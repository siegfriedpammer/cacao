\chapter{Class Loader}


\section{Introduction}

A \textit{Java Virtual Machine} (JVM) dynamically loads, links and
initializes classes and interfaces when they are needed. Loading a
class or interface means locating the binary representation---the
class files---and creating a class of interface structure from that
binary representation. Linking takes a loaded class or interface and
transfers it into the runtime state of the \textit{Java Virtual
Machine} so that it can be executed. Initialization of a class or
interface means executing the static class of interface initializer
\texttt{<clinit>}.

The following sections describe the process of loading, linking and
initalizing a class or interface in the CACAO \textit{Java Virtual
Machine} in greater detail. Further the used data structures and
techniques used in CACAO and the interaction with the GNU classpath
are described.


\section{System class loader}
\label{sectionsystemclassloader}

The class loader of a \textit{Java Virtual Machine} (JVM) is
responsible for loading all type of classes and interfaces into the
runtime system of the JVM. Every JVM has a \textit{system class
loader} which is implemented in \texttt{java.lang.ClassLoader} and
this class interacts via native function calls with the JVM itself.

\begingroup
\tolerance 10000
The \textit{GNU classpath} implements the system class loader in
\texttt{gnu.java.lang.SystemClassLoader} which extends
\texttt{java.lang.ClassLoader} and interacts with the JVM. The
\textit{bootstrap class loader} is implemented in
\texttt{java.lang.ClassLoader} plus the JVM depended class
\texttt{java.lang.VMClassLoader}. \texttt{java.lang.VMClassLoader} is
the main class how the bootstrap class loader of the GNU classpath
interacts with the JVM. The main functions of this class is

\endgroup

\begin{verbatim}
        static final native Class loadClass(String name, boolean resolve)
          throws ClassNotFoundException;
\end{verbatim}

\begingroup
\tolerance 10000
This is a native function implemented in the CACAO JVM, which is
located in \texttt{nat/VMClassLoader.c} and calls the internal loader
functions of CACAO. If the \texttt{name} argument is \texttt{NULL}, a
new \texttt{java.lang.NullPointerException} is created and the
function returns \texttt{NULL}.

\endgroup

If the \texttt{name} is non-NULL a new UTF8 string of the class' name
is created in the internal \textit{symbol table} via

\begin{verbatim}
        utf *javastring_toutf(java_lang_String *string, bool isclassname);
\end{verbatim}

This function converts a \texttt{java.lang.String} string into the
internal used UTF8 string representation. \texttt{isclassname} tells
the function to convert any \texttt{.} (periods) found in the class
name into \texttt{/} (slashes), so the class loader can find the
specified class.

Then a new \texttt{classinfo} structure is created via the

\begin{verbatim}
        classinfo *class_new(utf *classname);
\end{verbatim}

function call. This function creates a unique representation of this
class, identified by its name, in the JVM's internal \textit{class
hashtable}. The newly created \texttt{classinfo} structure (see
figure~\ref{classinfostructure}) is initialized with correct values,
like \texttt{loaded = false;}, \texttt{linked = false;} and
\texttt{initialized = false;}. This guarantees a definite state of a
new class.

\begin{figure}
\begin{verbatim}
    struct classinfo {                /* class structure                          */
        ...
        s4          flags;            /* ACC flags                                */
        utf        *name;             /* class name                               */

        s4          cpcount;          /* number of entries in constant pool       */
        u1         *cptags;           /* constant pool tags                       */
        voidptr    *cpinfos;          /* pointer to constant pool info structures */

        classinfo  *super;            /* super class pointer                      */
        classinfo  *sub;              /* sub class pointer                        */
        classinfo  *nextsub;          /* pointer to next class in sub class list  */

        s4          interfacescount;  /* number of interfaces                     */
        classinfo **interfaces;       /* pointer to interfaces                    */

        s4          fieldscount;      /* number of fields                         */
        fieldinfo  *fields;           /* field table                              */

        s4          methodscount;     /* number of methods                        */
        methodinfo *methods;          /* method table                             */
        ...
        bool        initialized;      /* true, if class already initialized       */
        bool        initializing;     /* flag for the compiler                    */
        bool        loaded;           /* true, if class already loaded            */
        bool        linked;           /* true, if class already linked            */
        s4          index;            /* hierarchy depth (classes) or index       */
                                      /* (interfaces)                             */
        s4          instancesize;     /* size of an instance of this class        */
    #ifdef SIZE_FROM_CLASSINFO
        s4          alignedsize;      /* size of an instance, aligned to the      */
                                      /* allocation size on the heap              */
    #endif

        vftbl_t    *vftbl;            /* pointer to virtual function table        */

        methodinfo *finalizer;        /* finalizer method                         */

        u2          innerclasscount;  /* number of inner classes                  */
        innerclassinfo *innerclass;
        ...
        utf        *packagename;      /* full name of the package                 */
        utf        *sourcefile;       /* classfile name containing this class     */
        java_objectheader *classloader; /* NULL for bootstrap classloader         */
    };
\end{verbatim}
\caption{\texttt{classinfo} structure}
\label{classinfostructure}
\end{figure}

The next step is to actually load the class requested. Thus the main
loader function

\begin{verbatim}
        classinfo *class_load(classinfo *c);
\end{verbatim}

is called, which is a wrapper function to the real loader function

\begin{verbatim}
        classinfo *class_load_intern(classbuffer *cb);
\end{verbatim}

This wrapper function is required to ensure some requirements:

\begin{itemize}
 \item enter a monitor on the \texttt{classinfo} structure, so that
 only one thread can load the same class or interface at the same time

 \item check if the class or interface is \texttt{loaded}, if it is
 \texttt{true}, leave the monitor and return immediately

 \item measure the loading time if requested

 \item initialize the \texttt{classbuffer} structure with the actual
 class file data

 \item reset the \texttt{loaded} field of the \texttt{classinfo}
 structure to \texttt{false} amd remove the \texttt{classinfo}
 structure from the internal class hashtable if we got an error or
 exception during loading

 \item free any allocated memory

 \item leave the monitor
\end{itemize}

The \texttt{class\_load} function is implemented to be
\textit{reentrant}. This must be the case for the \textit{eager class
loading} algorithm implemented in CACAO (described in more detail in
section \ref{sectioneagerclassloading}). Furthermore this means that
serveral threads can load different classes or interfaces at the same
time on multiprocessor machines.

The \texttt{class\_load\_intern} functions preforms the actual loading
of the binary representation of the class or interface. During loading
some verifier checks are performed which can throw an error. This
error can be a \texttt{java.lang.ClassFormatError} or a
\texttt{java.lang.NoClassDefFoundError}. Some of these
\texttt{java.lang.ClassFormatError} checks are

\begin{itemize}
 \item \textit{Truncated class file} --- unexpected end of class file
 data

 \item \textit{Bad magic number} --- class file does not start with
 the magic bytes (\texttt{0xCAFEBABE})

 \item \textit{Unsupported major.minor version} --- the bytecode
 version of the given class file is not supported by the JVM
\end{itemize}

The actual loading of the bytes from the binary representation is done
via the \texttt{suck\_*} functions. These functions are

\begin{itemize}
 \item \texttt{suck\_u1}: load one \texttt{unsigned byte} (8 bit)

 \item \texttt{suck\_u2}: load two \texttt{unsigned byte}s (16 bit)

 \item \texttt{suck\_u4}: load four \texttt{unsigned byte}s (32 bit)

 \item \texttt{suck\_u8}: load eight \texttt{unsigned byte}s (64 bit)

 \item \texttt{suck\_float}: load four \texttt{byte}s (32 bit)
 converted into a \texttt{float} value

 \item \texttt{suck\_double}: load eight \texttt{byte}s (64 bit)
 converted into a \texttt{double} value

 \item \texttt{suck\_nbytes}: load \textit{n} bytes
\end{itemize}

Loading \texttt{signed} values is done via the
\texttt{suck\_s[1,2,4,8]} macros which cast the loaded bytes to
\texttt{signed} values. All these functions take a
\texttt{classbuffer} (see figure~\ref{classbufferstructure})
structure pointer as argument.

\begin{figure}[h]
\begin{verbatim}
        typedef struct classbuffer {
            classinfo *class;               /* pointer to classinfo structure     */
            u1        *data;                /* pointer to byte code               */
            s4         size;                /* size of the byte code              */
            u1        *pos;                 /* current read position              */
        } classbuffer;
\end{verbatim}
\caption{\texttt{classbuffer} structure}
\label{classbufferstructure}
\end{figure}

This \texttt{classbuffer} structure is filled with data via the

\begin{verbatim}
        classbuffer *suck_start(classinfo *c);
\end{verbatim}

function. This function tries to locate the class, specifed with the
\texttt{classinfo} structure, in the \texttt{CLASSPATH}. This can be
a plain class file in the filesystem or a file in a
\texttt{zip}/\texttt{jar} file. If the class file is found, the
\texttt{classbuffer} is filled with data collected from the class
file, including the class file size and the binary representation of
the class.

Before reading any byte of the binary representation with a
\texttt{suck\_*} function, the remaining bytes in the
\texttt{classbuffer} data array must be checked with the

\begin{verbatim}
        static inline bool check_classbuffer_size(classbuffer *cb, s4 len);
\end{verbatim}

function. If the remaining bytes number is less than the amount of the
bytes to be read, specified by the \texttt{len} argument, a
\texttt{java.lang.ClassFormatError} with the detail message
\textit{Truncated class file}---as mentioned before---is thrown.

The following subsections describe chronologically in greater detail
the individual loading steps of a class or interface from it's binary
representation.


\subsection{Constant pool loading}
\label{sectionconstantpoolloading}

The class' constant pool is loaded via

\begin{verbatim}
        static bool class_loadcpool(classbuffer *cb, classinfo *c);
\end{verbatim}

from the \texttt{constant\_pool} table in the binary representation of
the class of interface. The \texttt{classinfo} structure has two
pointers to arrays which contain the class' constant pool infos,
namely: \texttt{cptags} and \texttt{cpinfos}. \texttt{cptags} contains
the type of the constant pool entry. \texttt{cpinfos} contains a
pointer to the constant pool entry itself.

The constant pool needs to be parsed in two passes. In the first pass
the information loaded is saved in temporary structures, which are
further processed in the second pass, when the complete constant pool
has been traversed. Only when all constant pool entries have been
processed, every constant pool entry can be completely resolved, but
this resolving can only be done in a specific order:

\begin{enumerate}
 \item \texttt{CONSTANT\_Class}

 \item \texttt{CONSTANT\_String}

 \item \texttt{CONSTANT\_NameAndType}

 \item \texttt{CONSTANT\_Fieldref} \\ \texttt{CONSTANT\_Methodref} \\
 \texttt{CONSTANT\_InterfaceMethodref} --- these entries are combined
 into one structure
\end{enumerate}

The temporary structures which are used to \textit{forward} the data
from the first pass into the second, are shown in
figure~\ref{constantpoolstructures}.

\begin{figure}[h]
\begin{verbatim}
        /* CONSTANT_Class entries */
        typedef struct forward_class {
            struct forward_class *next;
            u2 thisindex;
            u2 name_index;
        } forward_class;

        /* CONSTANT_String */
        typedef struct forward_string {
            struct forward_string *next;
            u2 thisindex;
            u2 string_index;
        } forward_string;

        /* CONSTANT_NameAndType */
        typedef struct forward_nameandtype {
            struct forward_nameandtype *next;
            u2 thisindex;
            u2 name_index;
            u2 sig_index;
        } forward_nameandtype;

        /* CONSTANT_Fieldref, CONSTANT_Methodref or CONSTANT_InterfaceMethodref */
        typedef struct forward_fieldmethint {
            struct forward_fieldmethint *next;
            u2 thisindex;
            u1 tag;
            u2 class_index;
            u2 nameandtype_index;
        } forward_fieldmethint;
\end{verbatim}
\caption{temporary constant pool structures}
\label{constantpoolstructures}
\end{figure}

The following list describes how the constant pool entries, which need
two passes, are processed in the first pass.

\begin{itemize}

 \item \texttt{CONSTANT\_Class}

 \begin{itemize}
  \item create a new \texttt{forward\_class} structure

  \item add the \texttt{forward\_class} structure to the
  \texttt{forward\_classes} list

  \item store the current constant pool index into the
  \texttt{thisindex} field

  \item read the index of the class' name via \texttt{suck\_u2} and
  store it into the \texttt{name\_index} field

  \item increase the constant pool index by one
 \end{itemize}

 \item \texttt{CONSTANT\_String}

 \begin{itemize}
  \item create a new \texttt{forward\_string} structure

  \item add the \texttt{forward\_string} structure to the \texttt{forward\_strings} list

  \item store the current constant pool index into the \texttt{thisindex} field

  \item read the index of the UTF8 string via \texttt{suck\_u2} and store it into the \texttt{name\_index} field

  \item increase the constant pool index by one
 \end{itemize}

 \item \texttt{CONSTANT\_NameAndType}

 \begin{itemize}
  \item create a new \texttt{forward\_nameandtype} structure

  \item add the \texttt{forward\_nameandtype} structure to the
  \texttt{forward\_nameandtypes} list

  \item store the current constant pool index into the
  \texttt{thisindex} field

  \item read the index of the UTF8 string containing the name via
  \texttt{suck\_u2} and store it into the \texttt{name\_index} field

  \item read the index of the UTF8 string containing the field or
  method descriptor via \texttt{suck\_u2} and store it into the
  \texttt{sig\_index} field

  \item increase the constant pool index by one
 \end{itemize}

 \item \texttt{CONSTANT\_Fieldref} \\ \texttt{CONSTANT\_Methodref} \\
 \texttt{CONSTANT\_InterfaceMethodref}

 \begin{itemize}
  \item create a new \texttt{forward\_fieldmethint} structure

  \item add the \texttt{forward\_fieldmethint} structure to the
  \texttt{forward\_fieldmethints} list

  \item store the current constant pool index into the
  \texttt{thisindex} field

  \item store the current constant pool type into the \texttt{tag}
  field

  \item read the constant pool index of the \texttt{CONSTANT\_Class}
  entry that contains the declaration of the field or method via
  \texttt{suck\_u2} and store it into the \texttt{class\_index} field

  \item read the constant pool index of the
  \texttt{CONSTANT\_NameAndType} entry that contains the name and
  descriptor of the field or method and store it into the
  \texttt{nameandtype\_index} field

  \item increase the constant pool index by one

 \end{itemize}

\end{itemize}

The remaining constant pool types can be completely resolved in the
first pass and need no further processing. These types, including the
actions taken in the first pass, are as follows:

\begin{itemize}

 \item \texttt{CONSTANT\_Integer}

 \begin{itemize}

  \item create a new \texttt{constant\_integer} structure (see
  figure~\ref{constantintegerstructure})

  \item read a 4 byte \texttt{signed integer} value via
  \texttt{suck\_s4} from the binary representation

  \item store the value into the \texttt{value} field of the
  \texttt{constant\_integer} structure

  \item store the type \texttt{CONSTANT\_Integer} into \texttt{cptags}
  and the pointer to the \texttt{constant\_integer} structure into
  \texttt{cpinfos} at the appropriate index

  \item increase the constant pool index by one

 \end{itemize}

\begin{figure}[h]
\begin{verbatim}
        typedef struct {            /* Integer                                    */
            s4 value;
        } constant_integer;
\end{verbatim}
\caption{\texttt{constant\_integer} structure}
\label{constantintegerstructure}
\end{figure}

 \item \texttt{CONSTANT\_Float}

 \begin{itemize}

  \item create a new \texttt{constant\_float} structure (see
  figure~\ref{constantfloatstructure})

  \item read a 4 byte \texttt{float} value via \texttt{suck\_float}
  from the binary representation

  \item store the value into the \texttt{value} field of the
  \texttt{constant\_float} structure

  \item store the type \texttt{CONSTANT\_Float} into \texttt{cptags}
  and the pointer to the \texttt{constant\_float} structure into
  \texttt{cpinfos} at the appropriate index

  \item increase the constant pool index by one

 \end{itemize}

\begin{figure}[h]
\begin{verbatim}
        typedef struct {            /* Float                                      */
            float value;
        } constant_float;
\end{verbatim}
\caption{\texttt{constant\_float} structure}
\label{constantfloatstructure}
\end{figure}

 \item \texttt{CONSTANT\_Long}

 \begin{itemize}

  \item create a new \texttt{constant\_long} structure (see
  figure~\ref{constantlongstructure})

  \item read a 8 byte \texttt{signed long} value via \texttt{suck\_s8}
  from the binary representation

  \item store the value into the \texttt{value} field of the
  \texttt{constant\_long} structure

  \item store the type \texttt{CONSTANT\_Long} into \texttt{cptags}
  and the pointer to the \texttt{constant\_long} structure into
  \texttt{cpinfos} at the appropriate index

  \item increase the constant pool index by two

 \end{itemize}

\begin{figure}[h]
\begin{verbatim}
        typedef struct {            /* Long                                       */
            s8 value;
        } constant_long;
\end{verbatim}
\caption{\texttt{constant\_long} structure}
\label{constantlongstructure}
\end{figure}

 \item \texttt{CONSTANT\_Double}

 \begin{itemize}

  \item create a new \texttt{constant\_double} structure (see
  figure~\ref{constantdoublestructure})

  \item read a 8 byte \texttt{double} value via \texttt{suck\_double}
  from the binary representation

  \item store the value into the \texttt{value} field of the
  \texttt{constant\_double} structure

  \item store the type \texttt{CONSTANT\_Double} into \texttt{cptags}
  and the pointer to the \texttt{constant\_double} structure into
  \texttt{cpinfos} at the appropriate index

  \item increase the constant pool index by two

 \end{itemize}

\begin{figure}[h]
\begin{verbatim}
        typedef struct {            /* Double                                     */
            double value;
        } constant_double;
\end{verbatim}
\caption{\texttt{constant\_double} structure}
\label{constantdoublestructure}
\end{figure}

 \item \texttt{CONSTANT\_Utf8}

 \begin{itemize}

  \item read the length of the UTF8 string via \texttt{suck\_u2}

  \item store the type \texttt{CONSTANT\_Utf8} into \texttt{cptags} at
  the appropriate index

  \item create a new UTF8 string in the runtime environment of the
  Java Virtual Machine via \texttt{utf\_new\_intern}

  \item store the pointer of the newly created UTF8 string into
  \texttt{cpinfos} at the appropriate index

  \item skip \texttt{length} bytes in the binary representation of the
  class or interface via \texttt{skip\_nbytes}

  \item increase the constant pool index by one

 \end{itemize}

\end{itemize}

In the second pass, the references are resolved and the runtime
structures are created. In further detail this includes for

\begin{itemize}

 \item \texttt{CONSTANT\_Class}

  \begin{itemize}

   \item resolve the UTF8 name string from the class' constant pool

   \item store the type \texttt{CONSTANT\_Class} in \texttt{cptags} at
   the appropriate index

   \item create a class in the class hashtable with the UTF8 name

   \item store the pointer to the new class in \texttt{cpinfos} at the
   appropriate index

  \end{itemize}

 \item \texttt{CONSTANT\_String}

  \begin{itemize}

   \item resolve the UTF8 string of the referenced string from the
   class' constant pool

   \item store type \texttt{CONSTANT\_String} in \texttt{cptags} and
   store the UTF8 string pointer into \texttt{cpinfos} at the
   appropriate index

  \end{itemize}

 \item \texttt{CONSTANT\_NameAndType}

  \begin{itemize}

   \item create a new \texttt{constant\_nameandtype} structure (see
   figure~\ref{constantnameandtype})

   \item resolve the UTF8 name and description string of the field or
   method and store them into the \texttt{constant\_nameandtype}
   structure

   \item store type \texttt{CONSTANT\_NameAndType} into
  \texttt{cptags} and store a pointer to the
  \texttt{constant\_nameandtype} structure into \texttt{cpinfos}

  \end{itemize}

\begin{figure}[h]
\begin{verbatim}
        typedef struct {            /* NameAndType (Field or Method)       */
            utf *name;              /* field/method name                   */
            utf *descriptor;        /* field/method type descriptor string */
        } constant_nameandtype;
\end{verbatim}
\caption{\texttt{constant\_nameandtype} structure}
\label{constantnameandtype}
\end{figure}

 \item \texttt{CONSTANT\_Fieldref} \\
       \texttt{CONSTANT\_Methodref} \\
       \texttt{CONSTANT\_InterfaceMethodref}

  \begin{itemize}

   \item create a new \texttt{constant\_FMIref} structure (see
   figure~\ref{constantFMIref})

   \item resolve the referenced \texttt{constant\_nameandtype}
   structure which contains the name and descriptor resolved in a
   previous step and store the name and descriptor into the
   \texttt{constant\_FMIref} structure

   \item resolve the pointer of the referenced class which was created
   in a previous step and store the pointer of the class into the
   \texttt{constant\_FMIref} structure

   \item store the type of the current constant pool entry in
   \texttt{cptags} and store the pointer to \texttt{constant\_FMIref}
   in \texttt{cpinfos} at the appropriate index

  \end{itemize}

\begin{figure}[h]
\begin{verbatim}
        typedef struct {           /* Fieldref, Methodref and InterfaceMethodref    */
            classinfo *class;      /* class containing this field/method/interface  */
            utf       *name;       /* field/method/interface name                   */
            utf       *descriptor; /* field/method/interface type descriptor string */
        } constant_FMIref;
\end{verbatim}
\caption{\texttt{constant\_FMIref} structure}
\label{constantFMIref}
\end{figure}

\end{itemize}

Any UTF8 strings, \texttt{constant\_nameandtype} structures or
referenced classes are resolved with the

\begin{verbatim}
        voidptr class_getconstant(classinfo *c, u4 pos, u4 ctype);
\end{verbatim}

function. This functions checks for type equality and then returns the
requested \texttt{cpinfos} slot of the specified class.


\subsection{Interface loading}

Interface loading is very simple and straightforward. After reading
the number of interfaces, for every interface referenced, a
\texttt{u2} constant pool index is read from the currently loading
class or interface. This index is used to resolve the interface class
via the \texttt{class\_getconstant} function from the class' constant
pool. This means, interface \textit{loading} is more interface
\textit{resolving} than loading. The resolved interfaces are stored
in an \texttt{classinfo *} array allocated by the class loader. The
memory pointer of the array is assigned to the \texttt{interfaces}
field of the \texttt{clasinfo} structure.


\subsection{Field loading}

The number of fields of the class or interface is read as \texttt{u2}
value. For each field the function

\begin{verbatim}
        static bool field_load(classbuffer *cb, classinfo *c, fieldinfo *f);
\end{verbatim}

is called. The \texttt{fieldinfo *} argument is a pointer to a
\texttt{fieldinfo} structure (see figure~\ref{fieldinfostructure})
allocated by the class loader. The fields' \texttt{name} and
\texttt{descriptor} are resolved from the class constant pool via
\texttt{class\_getconstant}. If the verifier option is turned on, the
fields' \texttt{flags}, \texttt{name} and \texttt{descriptor} are
checked for validity and can result in a
\texttt{java.lang.ClassFormatError}.

\begin{figure}[h]
\begin{verbatim}
    struct fieldinfo {        /* field of a class                                 */
        s4   flags;           /* ACC flags                                        */
        s4   type;            /* basic data type                                  */
        utf *name;            /* name of field                                    */
        utf *descriptor;      /* JavaVM descriptor string of field                */
	
        s4   offset;          /* offset from start of object (instance variables) */

        imm_union  value;     /* storage for static values (class variables)      */

        classinfo *class;     /* needed by typechecker. Could be optimized        */
                              /* away by using constant_FMIref instead of         */
                              /* fieldinfo throughout the compiler.               */
        ...
    };
\end{verbatim}
\caption{\texttt{fieldinfo} structure}
\label{fieldinfostructure}
\end{figure}

Each field can have some attributes. The number of attributes is read
as \texttt{u2} value from the binary representation. If the field has
the \texttt{ACC\_FINAL} bit set in the flags, the
\texttt{ConstantValue} attribute is available. This is the only
attribute processed by \texttt{field\_load} and can occur only once,
otherwise a \texttt{java.lang.ClassFormatError} is thrown. The
\texttt{ConstantValue} entry in the constant pool contains the value
for the \texttt{final} field. Depending on the fields' type, the
proper constant pool entry is resolved and assigned.


\subsection{Method loading}

As for the fields, the number of the class or interface methods is read from
the binary representation as \texttt{u2} value. For each method the function

\begin{verbatim}
        static bool method_load(classbuffer *cb, classinfo *c, methodinfo *m);
\end{verbatim}

is called. The beginning of the method loading code is nearly the same
as the field loading code. The \texttt{methodinfo *} argument is a
pointer to a \texttt{methodinfo} structure allocated by the class
loader. The method's \texttt{name} and \texttt{descriptor} are
resolved from the class constant pool via
\texttt{class\_getconstant}. With the verifier turned on, some method
checks are carried out. These include \texttt{flags}, \texttt{name}
and \texttt{descriptor} checks and argument count check.

\begin{figure}[h]
\begin{verbatim}
    struct methodinfo {                 /* method structure                       */
        java_objectheader header;       /* we need this in jit's monitorenter     */
        s4          flags;              /* ACC flags                              */
        utf        *name;               /* name of method                         */
        utf        *descriptor;         /* JavaVM descriptor string of method     */
        ...
        bool        isleafmethod;       /* does method call subroutines           */

        classinfo  *class;              /* class, the method belongs to           */
        s4          vftblindex;         /* index of method in virtual function    */
                                        /* table (if it is a virtual method)      */
        s4          maxstack;           /* maximum stack depth of method          */
        s4          maxlocals;          /* maximum number of local variables      */
        s4          jcodelength;        /* length of JavaVM code                  */
        u1         *jcode;              /* pointer to JavaVM code                 */
        ...
        s4          exceptiontablelength;/* exceptiontable length                 */
        exceptiontable *exceptiontable; /* the exceptiontable                     */

        u2          thrownexceptionscount;/* number of exceptions attribute       */
        classinfo **thrownexceptions;   /* checked exceptions a method may throw  */

        u2          linenumbercount;    /* number of linenumber attributes        */
        lineinfo   *linenumbers;        /* array of lineinfo items                */
        ...
        u1         *stubroutine;        /* stub for compiling or calling natives  */
        ...
    };
\end{verbatim}
\caption{\texttt{methodinfo} structure}
\label{methodinfostructure}
\end{figure}

The method loading function has to distinguish between a
\texttt{native} and a ''normal'' JAVA method. Depending on the
\texttt{ACC\_NATIVE} flags, a different stub is created.

For a JAVA method, a \textit{compiler stub} is created. The purpose of
this stub is to call the CACAO jit compiler with a pointer to the byte
code of the JAVA method as argument to compile the method into machine
code. During code generation a pointer to this compiler stub routine
is used as a temporary method call, if the method is not compiled
yet. After the target method is compiled, the new entry point of the
method is patched into the generated code and the compiler stub is
needless, thus it is freed.

If the method is a \texttt{native} method, the loader tries to find
the native function. If the function was found, a \textit{native stub}
is generated. This stub is responsible to manipulate the method's
arguments to be suitable for the \texttt{native} method called. This
includes inserting the \textit{JNI environment} pointer as first
argument and, if the \texttt{native} method has the
\texttt{ACC\_STATIC} flag set, inserting a pointer to the methods
class as second argument. If the \texttt{native} method is
\texttt{static}, the native stub also checks if the method's class is
already initialized. If the method's class is not initialized as the
native stub is generated, a \texttt{asm\_check\_clinit} calling code
is emitted.

Each method can have some attributes. The method loading function
processes two of them: \texttt{Code} and \texttt{Exceptions}.

The \texttt{Code} attribute is a \textit{variable-length} attribute
which contains the Java Virtual Machine instructions---the byte
code---of the JAVA method. If the method is either \texttt{native} or
\texttt{abstract}, it must not have a \texttt{Code} attribute,
otherwise it must have exactly one \texttt{Code}
attribute. Additionally to the byte code, the \texttt{Code} attribute
contains the exception table and attributes to \texttt{Code} attribute
itself. One exception table entry contains the \texttt{start\_pc},
\texttt{end\_pc} and
\texttt{handler\_pc} of the \texttt{try-catch} block, each read as
\texttt{u2} value, plus a reference to the class of the
\texttt{catch\_type}. Currently there are two attributes of the
\texttt{Code} attribute defined in the JVM specification:
\texttt{LineNumberTable} and \texttt{LocalVariableTable}. CACAO only
processes the \texttt{LineNumberTable} attribute. A
\texttt{LineNumberTable} entry consist of the \texttt{start\_pc} and
the \texttt{line\_number}, which are stored in a \texttt{lineinfo}
structure (see figure~\ref{lineinfostructure}).

\begin{figure}[h]
\begin{verbatim}
    struct lineinfo {
        u2 start_pc;
        u2 line_number;
    };
\end{verbatim}
\caption{\texttt{lineinfo} structure}
\label{lineinfostructure}
\end{figure}

The linenumber count and the memory pointer of the \texttt{lineinfo}
structure array are assigned to the \texttt{classinfo} fields
\texttt{linenumbercount} and \texttt{linenumbers} respectively.

The \texttt{Exceptions} attribute is a \textit{variable-length}
attribute and contains the checked exceptions the JAVA method may
throw. The \texttt{Exceptions} attribute consist of the count of
exceptions, which is stored in the \texttt{classinfo} field
\texttt{thrownexceptionscount}, and the adequate amount of \texttt{u2}
constant pool index values. The exception classes are resolved from
the constant pool and stored in an allocated \texttt{classinfo *}
array, whose memory pointer is assigned to the
\texttt{thrownexceptions} field of the \texttt{classinfo} structure.

Any attributes which are not processed by the CACAO class loading
system, are skipped via

\begin{verbatim}
        static bool skipattributebody(classbuffer *cb);
\end{verbatim}

which skips one attribute or

\begin{verbatim}
        static bool skipattributes(classbuffer *cb, u4 num);
\end{verbatim}

which skips a specified number \texttt{num} of attributes. If any
problem occurs in the method loading function, a
\texttt{java.lang.ClassFormatError} with a specific detail message is
thrown.


\subsection{Attribute loading}

Attribute loading is done via the

\begin{verbatim}
        static bool attribute_load(classbuffer *cb, classinfo *c, u4 num);
\end{verbatim}

function. The currently loading class or interface can contain some
additional attributes which have not already been loaded. The CACAO
system class loader processes two of them: \texttt{InnerClasses} and
\texttt{SourceFile}.

The \texttt{InnerClass} attribute is a \textit{variable-length}
attribute in the \texttt{attributes} table of the binary
representation of the class or interface. A \texttt{InnerClass} entry
contains the \texttt{inner\_class} constant pool index itself, the
\texttt{outer\_class} index, the \texttt{name} index of the inner
class' name and the inner class' \texttt{flags} bitmask. All these
values are read in \texttt{u2} chunks.

The constant pool indexes are used with the

\begin{verbatim}
        voidptr innerclass_getconstant(classinfo *c, u4 pos, u4 ctype);
\end{verbatim}

function call to resolve the classes or UTF8 strings. After resolving
is done, all values are stored in the \texttt{innerclassinfo}
structure (see figure~\ref{innerclassinfostructure}).

\begin{figure}[h]
\begin{verbatim}
    struct innerclassinfo {
        classinfo *inner_class;       /* inner class pointer                      */
        classinfo *outer_class;       /* outer class pointer                      */
        utf       *name;              /* innerclass name                          */
        s4         flags;             /* ACC flags                                */
    };
\end{verbatim}
\caption{\texttt{innerclassinfo} structure}
\label{innerclassinfostructure}
\end{figure}

The other attribute, \texttt{SourceFile}, is just one \texttt{u2}
constant pool index value to get the UTF8 string reference of the
class' \texttt{SourceFile} name. The string pointer is assigned to the
\texttt{sourcefile} field of the \texttt{classinfo} structure.

Both attributes must occur only once. Other attributes than these two
are skipped with the earlier mentioned \texttt{skipattributebody}
function.

After the attribute loading is done and no error occured, the
\texttt{class\_load\_intern} function returns the \texttt{classinfo}
pointer to signal that there was no problem. If \texttt{NULL} is
returned, there was an exception.


%\section{Dynamic class loader}


\section{Eager - lazy class loading}

A Java Virtual Machine can implement two different algorithms for the
system class loader to load classes or interfaces: \textit{eager class
loading} and \textit{lazy class loading}.


\subsection{Eager class loading}
\label{sectioneagerclassloading}

The Java Virtual Machine initially creates, loads and links the class
of the main program with the system class loader. The creation of the
class is done via the \texttt{class\_new} function call (see section
\ref{sectionsystemclassloader}). In this function, with \textit{eager
loading} enabled, firstly the currently created class or interface is
loaded with \texttt{class\_load}. CACAO uses the \textit{eager class
loading} algorithm with the command line switch \texttt{-eager}. As
described in the ''Constant pool loading'' section (see
\ref{sectionconstantpoolloading}), the binary representation of a
class or interface contains references to other classes or
interfaces. With \textit{eager loading} enabled, referenced classes or
interfaces are loaded immediately.

If a class reference is found in the second pass of the constant pool
loading process, the class is created in the class hashtable with
\texttt{class\_new\_intern}. CACAO uses the intern function here
because the normal \texttt{class\_new} function, which is a wrapper
function, instantly tries to \textit{link} all referenced
classes. This must not happen until all classes or interfaces
referenced are loaded, otherwise the Java Virtual Machine gets into an
indefinite state.

After the \texttt{classinfo} of the class referenced is created, the
class or interface is \textit{loaded} via the \texttt{class\_load}
function (described in more detail in section
\ref{sectionsystemclassloader}). When the class loading function
returns, the current referenced class or interface is added to a list
called \texttt{unlinkedclasses}, which contains all loaded but
unlinked classes referenced by the currently loaded class or
interface. This list is processed in the \texttt{class\_new} function
of the currently created class or interface after \texttt{class\_load}
returns. For each entry in the \texttt{unlinkedclasses} list,
\texttt{class\_link} is called which finally \textit{links} the class
(described in more detail in section \ref{sectionlinking}) and then
the class entry is removed from the list. When all referenced classes
or interfaces are linked, the currently created class or interface is
linked and the \texttt{class\_new} functions returns.


\subsection{Lazy class loading}
\label{sectionlazyclassloading}

Usually it takes much more time for a Java Virtual Machine to start a
program with \textit{eager class loading} as with \textit{lazy class
loading}. With \textit{eager class loading}, a typical
\texttt{HelloWorld} program needs 513 class loads with the current GNU
classpath CACAO is using. When using \textit{lazy class loading},
CACAO only needs 121 class loads for the same \texttt{HelloWorld}
program. This means with \textit{lazy class loading} CACAO needs to
load more than four times less class files. Furthermore CACAO does
also \textit{lazy class linking}, which saves much more run-time here.

CACAO's \textit{lazy class loading} implementation does not completely
follow the JVM specification. A Java Virtual Machine which implements
\textit{lazy class loading} should load and link requested classes or
interfaces at runtime. But CACAO does class loading and linking at
parse time, because of some problems not resolved yet. That means, if
a Java Virtual Machine instruction is parsed which uses any class or
interface references, like \texttt{JAVA\_PUTSTATIC},
\texttt{JAVA\_GETFIELD} or any \texttt{JAVA\_INVOKE*} instructions,
the referenced class or interface is loaded and linked immediately
during the parse pass of currently compiled method. This introduces
some incompatibilities with other Java Virtual Machines like Sun's
JVM, IBM's JVM or Kaffe.

Given a code snippet like this

\begin{verbatim}
        void sub(boolean b) {
            if (b) {
                new A();
            }
            System.out.println("foobar");
        }
\end{verbatim}

If the function is called with \texttt{b} equal \texttt{false} and the
class file \texttt{A.class} does not exist, a Java Virtual Machine
should execute the code without any problems, print \texttt{foobar}
and exit the Java Virtual Machine with exit code 0. Due to the fact
that CACAO does class loading and linking at parse time, the CACAO
Virtual Machine throws an \texttt{java.lang.NoClassDefFoundError:~A}
exception which is not caught and thus discontinues the execution
without printing \texttt{foobar} and exits.

The CACAO development team has not yet a solution for this
problem. It's not trivial to move the loading and linking process from
the compilation phase into runtime, especially CACAO was initially
designed for \textit{eager class loading} and \textit{lazy class
loading} was implemented at a later time to optimize class loading and
to get a little closer to the JVM specification. \textit{Lazy class
loading} at runtime is one of the most important features to be
implemented in the future. It is essential to make CACAO a standard
compliant Java Virtual Machine.


\section{Linking}
\label{sectionlinking}

Linking is the process of preparing a previously loaded class or
interface to be used in the Java Virtual Machine's runtime
environment. The function which performs the linking in CACAO is

\begin{verbatim}
        classinfo *class_link(classinfo *c);
\end{verbatim}

This function, as for class loading, is just a wrapper function to the
main linking function

\begin{verbatim}
        static classinfo *class_link_intern(classinfo *c);
\end{verbatim}

This function should not be called directly and is thus declared as
\texttt{static}. The purposes of the wrapper function are

\begin{itemize}
 \item enter a monitor on the \texttt{classinfo} structure, so that
 only one thread can link the same class or interface at the same time

 \item check if the class or interface is \texttt{linked}, if it is
 \texttt{true}, leave the monitor and return immediately

 \item measure linking time if requested

 \item check if the intern linking function has thrown an error or an
 exception and reset the \texttt{linked} field of the
 \texttt{classinfo} structure

 \item leave the monitor
\end{itemize}

The \texttt{class\_link} function, like the \texttt{class\_load}
function, is implemented to be \textit{reentrant}. This must be the
case for the linking algorithm implemented in CACAO. Furthermore this
means that serveral threads can link different classes or interfaces
at the same time on multiprocessor machines.

The first step in the \texttt{class\_link\_intern} function is to set
the \texttt{linked} field of the currently linked \texttt{classinfo}
structure to \texttt{true}. This is essential, that the linker does
not try to link a class or interface again, while it's already in the
linking process. Such a case can occur because the linker also
processes the class' direct superclass and direct superinterfaces.

In CACAO's linker the direct superinterfaces are processed first. For
each interface in the \texttt{interfaces} field of the
\texttt{classinfo} structure is checked if there occured an
\texttt{java.lang.ClassCircularityError}, which happens when the
currently linked class or interface is equal the interface which
should be processed. Otherwise the interface is loaded and linked if
not already done. After the interface is loaded successfully, the
interface flags are checked for the \texttt{ACC\_INTERFACE} bit. If
this is not the case, a
\texttt{java.lang.IncompatibleClassChangeError} is thrown and
\texttt{class\_link\_intern} returns.

Then the direct superclass is handled. If the direct superclass is
equal \texttt{NULL}, we have the special case of linking
\texttt{java.lang.Object}. There are only set some \texttt{classinfo}
fields to special values for \texttt{java.lang.Object} like

\begin{verbatim}
        c->index = 0;
        c->instancesize = sizeof(java_objectheader);
        vftbllength = 0;
        c->finalizer = NULL;
\end{verbatim}

If the direct superclass is non-\texttt{NULL}, CACAO firstly detects
class circularity as for interfaces. If no
\texttt{java.lang.ClassCircularityError} was thrown, the superclass is
loaded and linked if not already done before. Then some flag bits of
the superclass are checked: \texttt{ACC\_INTERFACE} and
\texttt{ACC\_FINAL}. If one of these bits is set an error is thrown.

If the currently linked class is an array, CACAO calls a special array
linking function

\begin{verbatim}
        static arraydescriptor *class_link_array(classinfo *c);
\end{verbatim}

This function firstly checks if the passed \texttt{classinfo} is an
\textit{array of arrays} or an \textit{array of objects}. In both
cases the component type is created in the class hashtable via
\texttt{class\_new} and then loaded and linked if not already
done. If none is the case, the passed array is a \textit{primitive
type array}. No matter of which type the array is, an
\texttt{arraydescriptor} structure (see
figure~\ref{arraydescriptorstructure}) is allocated and filled with
the appropriate values of the given array type.

\begin{figure}[h]
\begin{verbatim}
    struct arraydescriptor {
        vftbl_t *componentvftbl; /* vftbl of the component type, NULL for primit. */
        vftbl_t *elementvftbl;   /* vftbl of the element type, NULL for primitive */
        s2       arraytype;      /* ARRAYTYPE_* constant                          */
        s2       dimension;      /* dimension of the array (always >= 1)          */
        s4       dataoffset;     /* offset of the array data from object pointer  */
        s4       componentsize;  /* size of a component in bytes                  */
        s2       elementtype;    /* ARRAYTYPE_* constant                          */
    };
\end{verbatim}
\caption{\texttt{arraydescriptor} structure}
\label{arraydescriptorstructure}
\end{figure}

After the \texttt{class\_link\_array} function call, the class
\texttt{index} is calculated. For interfaces---classes with
\texttt{ACC\_INTERFACE} flag bit set---the class' \texttt{index} is
the global \texttt{interfaceindex} plus one. Any other classes get the
\texttt{index} of the superclass plus one.

Other \texttt{classinfo} fields are also set from the superclass like,
\texttt{instancesize}, \texttt{vftbllength} and the \texttt{finalizer}
function. All these values are temporary ones and can be overwritten
at a later time.

The next step in \texttt{class\_load\_intern} is to compute the
\textit{virtual function table length}. For each method in
\texttt{classinfo}'s \texttt{methods} field which has not the
\texttt{ACC\_STATIC} flag bit set, thus is an instance method, the
direct superclasses up to \texttt{java.lang.Object} are checked with

\begin{verbatim}
        static bool method_canoverwrite(methodinfo *m, methodinfo *old);
\end{verbatim}

if the current method can overwrite the superclass method, if there
exists one. If the found superclass method has the
\texttt{ACC\_PRIVATE} flag bit set, the current method's
\textit{virtual function table index} is the current \textit{virtual
function table length} plus one:

\begin{verbatim}
        m->vftblindex = (vftbllength++);
\end{verbatim}

If the current method has the \texttt{ACC\_FINAL} flag bit set, the
CACAO class linker throws a \texttt{java.lang.VerifyError}. Otherwise
the current method's \textit{virtual function table index} is the same
as the index from the superclass method:

\begin{verbatim}
        m->vftblindex = tc->methods[j].vftblindex;
\end{verbatim}

After processing the \textit{virtual function table length}, the CACAO
linker computes the \textit{interface table length}. For the current
class' and every superclass' interfaces, the function

\begin{verbatim}
        static s4 class_highestinterface(classinfo *c);
\end{verbatim}

is called. This function computes the highest interface \texttt{index}
of the passed interface and returns the value. This is done by
recursively calling \texttt{class\_highestinterface} with each
interface from the \texttt{interfaces} array of the passed interface
as argument. The highest \texttt{index} value found is the
\textit{interface table length} of the currently linking class or
interface.

Now that the linker has completely computed the size of the
\textit{virtual function table}, the memory can be allocated, casted
to an \texttt{vftbl} structure (see figure~\ref{vftblstructure}) and
filled with the previously calculated values.

\begin{figure}
\begin{verbatim}
    struct vftbl {
        methodptr   *interfacetable[1];    /* interface table (access via macro)  */

        classinfo   *class;                /* class, the vtbl belongs to          */

        arraydescriptor *arraydesc;        /* for array classes, otherwise NULL   */

        s4           vftbllength;          /* virtual function table length       */
        s4           interfacetablelength; /* interface table length              */

        s4           baseval;              /* base for runtime type check         */
                                           /* (-index for interfaces)             */
        s4           diffval;              /* high - base for runtime type check  */

        s4          *interfacevftbllength; /* length of interface vftbls          */
	
        methodptr    table[1];             /* class vftbl                         */
    };
\end{verbatim}
\caption{\texttt{vftbl} structure}
\label{vftblstructure}
\end{figure}

Some important values are

\begin{verbatim}
        c->header.vftbl = c->vftbl = v;
        v->class = c;
        v->vftbllength = vftbllength;
        v->interfacetablelength = interfacetablelength;
        v->arraydesc = arraydesc;
\end{verbatim}

If the currently linked class is an interface, the \texttt{baseval} of
the interface's \textit{virtual function table} is set to
\texttt{-(c->index)}. Then the \textit{virtual function table} of the
direct superclass is copied into the \texttt{table} field of the
current \textit{virtual function table} and for each
non-\texttt{static} method in the current's class or interface
\texttt{methods} field, the pointer to the \textit{stubroutine} of the
method in stored in the \textit{virtual function table}.

Now the fields of the currently linked class or interface are
processed. The CACAO linker computes the instance size of the class or
interface and the offset of each field inside. For each field in the
\texttt{classinfo} field \texttt{fields} which is non-\texttt{static},
the type-size is resolved via the \texttt{desc\_typesize} function
call. Then a new \texttt{instancesize} is calculated with

\begin{verbatim}
        c->instancesize = ALIGN(c->instancesize, dsize);
\end{verbatim}

which does memory alignment suitable for the next field. This newly
computed \texttt{instancesize} is the \texttt{offset} of the currently
processed field. The type-size is then added to get the real
\texttt{instancesize}.

The next step of the CACAO linker is to initialize two \textit{virtual
function table} fields, namely \texttt{interfacevftbllength} and
\texttt{interfacetable}. For \texttt{interfacevftbllength} an
\texttt{s4} array of \texttt{interfacetablelength} elements is
allocated. Each \texttt{interfacevftbllength} element is initialized
with \texttt{0} and the elements in \texttt{interfacetable} with
\texttt{NULL}. After the initialization is done, the interfaces of the
currently linked class and all it's superclasses, up to
\texttt{java.lang.Object}, are processed via the

\begin{verbatim}
        static void class_addinterface(classinfo *c, classinfo *ic);
\end{verbatim}

function call. This function adds the methods of the passed interface
to the \textit{virtual function table} of the passed class or
interface. If the method count of the passed interface is zero, the
function adds a method fake entry, which is needed for subtype
tests:

\begin{verbatim}
        v->interfacevftbllength[i] = 1;
        v->interfacetable[-i] = MNEW(methodptr, 1);
        v->interfacetable[-i][0] = NULL;
\end{verbatim}

\texttt{i} represents the \texttt{index} of the passed interface
\texttt{ic}, \texttt{v} the \textit{virtual function table} of the
passed class or interface \texttt{c}.

If the method count is non-zero, an \texttt{methodptr} array of
\texttt{ic->methodscount} elements is allocated and the method count
value is stored in the particular position of the
\texttt{interfacevftbllength} array:

\begin{verbatim}
        v->interfacevftbllength[i] = ic->methodscount;
        v->interfacetable[-i] = MNEW(methodptr, ic->methodscount);
\end{verbatim}

For each method of the passed interface, the methods of the passed
target class or interface and all superclass methods, up to
\texttt{java.lang.Object}, are checked if they can overwrite the
interface method via \texttt{method\_canoverwrite}. If the function
returns \texttt{true}, the corresponding function is resolved from the
\texttt{table} field of the \textit{virtual function table} and stored
it the particular position of the \texttt{interfacetable}:

\begin{verbatim}
        v->interfacetable[-i][j] = v->table[mi->vftblindex];
\end{verbatim}

The \texttt{class\_addinterface} function is also called recursively
for all interfaces the interface passed implements.

After the interfaces were added and the currently linked class or
interface is not \texttt{java.lang.Object}, the CACAO linker tries to
find a function which name and descriptor matches
\texttt{finalize()V}. If an appropriate function was found and the
function is non-\texttt{static}, it is assigned to the
\texttt{finalizer} field of the \texttt{classinfo} structure. CACAO
does not assign the \texttt{finalize()V} function to
\texttt{java.lang.Object}, because this function is inherited to all
subclasses which do not explicitly implement a \texttt{finalize()V}
method. This would mean, for each instantiated object, which is marked
for collection in the Java Virtual Machine, an empty function would be
called from the garbage collector when a garbage collection takes
place.

The final task of the linker is to compute the \texttt{baseval} and
\texttt{diffval} values from the subclasses of the currently linked
class or interface. These values are used for \textit{runtime type
checking} (described in more detail in
section~\ref{sectionruntimetypechecking}). The calculation is done via
the

\begin{verbatim}
        void loader_compute_subclasses(classinfo *c);
\end{verbatim}

function call. This function sets the \texttt{nextsub} and
\texttt{sub} fields of the \texttt{classinfo} structure, resets the
global \texttt{classvalue} variable to zero and calls the

\begin{verbatim}
        static void loader_compute_class_values(classinfo *c);
\end{verbatim}

function with \texttt{java.lang.Object} as parameter. First of the
all, the \texttt{baseval} is set of the currently passed class or
interface. The \texttt{baseval} is the global \texttt{classvalue}
variable plus one:

\begin{verbatim}
        c->vftbl->baseval = ++classvalue;
\end{verbatim}

Then all subclasses of the currently passed class or interface are
processed. For each subclass found,
\texttt{loader\_compute\_class\_values} is recursively called. After
all subclasses have been processed, the \texttt{diffval} of the
currently passed class or interface is calculated. It is the
difference of the current global \texttt{classvalue} variable value
and the previously \texttt{baseval} set:

\begin{verbatim}
        c->vftbl->diffval = classvalue - c->vftbl->baseval;
\end{verbatim}

After the \texttt{baseval} and \texttt{diffval} values are newly
calculated for all classes and interfaces in the Java Virtual Machine,
the internal linker function \texttt{class\_link\_intern} returns the
currently linking \texttt{classinfo} structure pointer, to indicate
that the linker function did not raise an error or exception.


\section{Initialization}
\label{sectioninitialization}

A class or interface can have a \texttt{static} initialization
function called \textit{static class initializer}. The function has
the name \texttt{<clinit>()V}. This function must be invoked before a
\texttt{static} function of the class is called or a \texttt{static}
field is accessed via \texttt{ICMD\_PUTSTATIC} or
\texttt{ICMD\_GETSTATIC}. In CACAO

\begin{verbatim}
        classinfo *class_init(classinfo *c);
\end{verbatim}

is responsible for the invocation of the \textit{static class
initializer}. It is, like for class loading and class linking, just a
wrapper function to the main initializing function

\begin{verbatim}
        static classinfo *class_init_intern(classinfo *c);
\end{verbatim}

The wrapper function has the following purposes:

\begin{itemize}
 \item enter a monitor on the \texttt{classinfo} structure, so that
 only one thread can initialize the same class or interface at the
 same time

 \item check if the class or interface is \texttt{initialized} or
 \texttt{initializing}, if one is \texttt{true}, leave the monitor and
 return

 \item tag the class or interface as \texttt{initializing}

 \item call the internal initialization function
 \texttt{class\_init\_intern}

 \item if the internal initialization function returns
 non-\texttt{NULL}, the class or interface is tagged as
 \texttt{initialized}

 \item reset the \texttt{initializing} flag

 \item leave the monitor
\end{itemize}

The intern initializing function should not be called directly,
because of race conditions of concurrent threads. Two or more
different threads could access a \texttt{static} field or call a
\texttt{static} function of an uninitialized class at almost the same
time. This means that each single thread would invoke the
\textit{static class initializer} and this would lead into some
problems.

The CACAO initializer needs to tag the class or interface as currently
initializing. This is done by setting the \texttt{initializing} field
of the \texttt{classinfo} structure to \texttt{true}. CACAO needs this
field in addition to the \texttt{initialized} field for two reasons:

\begin{itemize}
 \item Another concurrently running thread can access a
 \texttt{static} field of the currently initializing class or
 interface. If the class or interface of the \texttt{static} field was
 not initialized during code generation, some special code was
 generated for the \texttt{ICMD\_PUTSTATIC} and
 \texttt{ICMD\_GETSTATIC} intermediate commands. This special code is
 a call to an architecture dependent assembler function named
 \texttt{asm\_check\_clinit}. Since this function is speed optimized
 for the case that the target class is already initialized, it only
 checks for the \texttt{initialized} field and does not take care of
 any monitor that may have been entered. If the \texttt{initialized}
 flag is \texttt{false}, the assembler function calls the
 \texttt{class\_init} function where it probably stops at the monitor
 enter. Due to this fact, the thread which does the initialization can
 not set the \texttt{initialized} flag to \texttt{true} when the
 initialization starts, otherwise potential concurrently running
 threads would continue their execution although the \textit{static
 class initializer} has not finished yet.

 \item The thread which is currently \texttt{initializing} the class
 or interface can pass the monitor which has been entered and thus
 needs to know if this class or interface is currently initialized.
\end{itemize}

Firstly \texttt{class\_init\_intern} checks if the passed class or
interface is loaded and linked. If not, the particular action is
taken. This is just a safety measure, because---CACAO
internally---each class or interface should have been already loaded
and linked before \texttt{class\_init} is called.

Then the superclass, if any specified, is checked if it is already
initialized. If not, the initialization is done immediately. The same
check is performed for each interface in the \texttt{interfaces} array
of the \texttt{classinfo} structure of the current class or interface.

After the superclass and all interfaces are initialized, CACAO tries
to find the \textit{static class initializer} function, where the
method name matches \texttt{<clinit>} and the method descriptor
\texttt{()V}. If no \textit{static class initializer} method is found in the
current class or interface, the \texttt{class\_link\_intern} functions
returns immediately without an error. If a \textit{static class
initializer} method is found, it's called with the architecture
dependent assembler function \texttt{asm\_calljavafunction}.

Exception handling of an exception thrown in an \textit{static class
initializer} is a bit different than usual. It depends on the type of
exception. If the exception thrown is an instance of
\texttt{java.lang.Error}, the \texttt{class\_init\_intern} function
just returns \texttt{NULL}. If the exception thrown is an instance of
\texttt{java.lang.Exception}, the exception is wrapped into a
\texttt{java.lang.ExceptionInInitializerError}. This is done via the
\texttt{new\_exception\_throwable} function call. The newly generated
error is set as exception thrown and the \texttt{class\_init\_intern}
returns \texttt{NULL}.

If no exception occurred in the \textit{static class initializer}, the
internal initializing function returns the current \texttt{classinfo}
structure pointer to indicate, that the initialization was successful.
