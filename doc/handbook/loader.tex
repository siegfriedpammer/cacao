\chapter{Class Loader}


\section{Introduction}

A \textit{Java Virtual Machine} (JVM) dynamically loads, links and
initializes classes and interfaces when they are needed. Loading a
class or interface means locating the binary representation---the
class files---and creating a class of interface structure from that
binary representation. Linking takes a loaded class or interface and
transfers it into the runtime state of the \textit{Java Virtual
Machine} so that it can be executed. Initialization of a class or
interface means executing the static class of interface initializer
\texttt{<clinit>}.

The following sections describe the process of loading, linking and
initalizing a class or interface in the CACAO \textit{Java Virtual
Machine} in greater detail. Further the used data structures and
techniques used in CACAO and the interaction with the GNU classpath
are described.


\section{System class loader}
\label{sectionsystemclassloader}

The class loader of a \textit{Java Virtual Machine} (JVM) is
responsible for loading all type of classes and interfaces into the
runtime system of the JVM. Every JVM has a \textit{system class
loader} which is implemented in \texttt{java.lang.ClassLoader} and
this class interacts via native function calls with the JVM itself.

\begingroup
\tolerance 10000
The \textit{GNU classpath} implements the system class loader in
\texttt{gnu.java.lang.SystemClassLoader} which extends
\texttt{java.lang.ClassLoader} and interacts with the JVM. The
\textit{bootstrap class loader} is implemented in
\texttt{java.lang.ClassLoader} plus the JVM depended class
\texttt{java.lang.VMClassLoader}. \texttt{java.lang.VMClassLoader} is
the main class how the bootstrap class loader of the GNU classpath
interacts with the JVM. The main functions of this class is

\endgroup

\begin{verbatim}
        static final native Class loadClass(String name, boolean resolve)
          throws ClassNotFoundException;
\end{verbatim}

\begingroup
\tolerance 10000
This is a native function implemented in the CACAO JVM, which is
located in \texttt{nat/VMClassLoader.c} and calls the internal loader
functions of CACAO. If the \texttt{name} argument is \texttt{NULL}, a
new \texttt{java.lang.NullPointerException} is created and the
function returns \texttt{NULL}.

\endgroup

If the \texttt{name} is non-NULL a new UTF8 string of the class' name
is created in the internal \textit{symbol table} via

\begin{verbatim}
        utf *javastring_toutf(java_lang_String *string, bool isclassname);
\end{verbatim}

This function converts a \texttt{java.lang.String} string into the
internal used UTF8 string representation. \texttt{isclassname} tells
the function to convert any \texttt{.} (periods) found in the class
name into \texttt{/} (slashes), so the class loader can find the
specified class.

Then a new \texttt{classinfo} structure is created via the

\begin{verbatim}
        classinfo *class_new(utf *classname);
\end{verbatim}

function call. This function creates a unique representation of this
class, identified by its name, in the JVM's internal \textit{class
hashtable}. The newly created \texttt{classinfo} structure (Figure
\ref{classinfostructure}) is initialized with correct values, like
\texttt{loaded = false;}, \texttt{linked = false;} and
\texttt{initialized = false;}. This guarantees a definite state of a
new class.

\begin{figure}
\begin{verbatim}
    struct classinfo {                /* class structure                          */
        ...
        s4          flags;            /* ACC flags                                */
        utf        *name;             /* class name                               */

        s4          cpcount;          /* number of entries in constant pool       */
        u1         *cptags;           /* constant pool tags                       */
        voidptr    *cpinfos;          /* pointer to constant pool info structures */

        classinfo  *super;            /* super class pointer                      */
        ...
        s4          interfacescount;  /* number of interfaces                     */
        classinfo **interfaces;       /* pointer to interfaces                    */

        s4          fieldscount;      /* number of fields                         */
        fieldinfo  *fields;           /* field table                              */

        s4          methodscount;     /* number of methods                        */
        methodinfo *methods;          /* method table                             */
        ...
        bool        initialized;      /* true, if class already initialized       */
        bool        initializing;     /* flag for the compiler                    */
        bool        loaded;           /* true, if class already loaded            */
        bool        linked;           /* true, if class already linked            */
        s4          index;            /* hierarchy depth (classes) or index       */
                                      /* (interfaces)                             */
        s4          instancesize;     /* size of an instance of this class        */
    #ifdef SIZE_FROM_CLASSINFO
        s4          alignedsize;      /* size of an instance, aligned to the      */
                                      /* allocation size on the heap              */
    #endif

        vftbl_t    *vftbl;            /* pointer to virtual function table        */

        methodinfo *finalizer;        /* finalizer method                         */

        u2          innerclasscount;  /* number of inner classes                  */
        innerclassinfo *innerclass;
        ...
        utf        *packagename;      /* full name of the package                 */
        utf        *sourcefile;       /* classfile name containing this class     */
        java_objectheader *classloader; /* NULL for bootstrap classloader         */
    };
\end{verbatim}
\caption{\texttt{classinfo} structure}
\label{classinfostructure}
\end{figure}

The next step is to actually load the class requested. Thus the main
loader function

\begin{verbatim}
        classinfo *class_load(classinfo *c);
\end{verbatim}

is called, which is a wrapper function to the real loader function

\begin{verbatim}
        classinfo *class_load_intern(classbuffer *cb);
\end{verbatim}

This wrapper function is required to ensure some requirements:

\begin{itemize}
 \item enter a monitor on the \texttt{classinfo} structure, so that
 only one thread can load the same class at the same time

 \item measure the loading time if requested

 \item initialize the \texttt{classbuffer} structure with the actual
 class file data

 \item reset the \texttt{loaded} field of the \texttt{classinfo}
 structure to \texttt{false} amd remove the \texttt{classinfo}
 structure from the internal class hashtable if we got an error or
 exception during loading

 \item free any allocated memory and leave the monitor
\end{itemize}

The \texttt{class\_load} function is implemented to be
\textit{reentrant}. This must be the case for the \textit{eager class
loading} algorithm implemented in CACAO (described in more detail in
section \ref{sectioneagerclassloading}). Furthermore this means that
serveral threads can load different classes or interfaces at the same
time on multiprocessor machines.

The \texttt{class\_load\_intern} functions preforms the actual loading
of the binary representation of the class or interface. During loading
some verifier checks are performed which can throw an error. This
error can be a \texttt{java.lang.ClassFormatError} or a
\texttt{java.lang.NoClassDefFoundError}. Some of these
\texttt{java.lang.ClassFormatError} checks are

\begin{itemize}
 \item \textit{Truncated class file} --- unexpected end of class file
 data

 \item \textit{Bad magic number} --- class file does not start with
 the magic bytes (\texttt{0xCAFEBABE})

 \item \textit{Unsupported major.minor version} --- the bytecode
 version of the given class file is not supported by the JVM
\end{itemize}

The actual loading of the bytes from the binary representation is done
via the \texttt{suck\_*} functions. These functions are

\begin{itemize}
 \item \texttt{suck\_u1}: load one \texttt{unsigned byte} (8 bit)

 \item \texttt{suck\_u2}: load two \texttt{unsigned byte}s (16 bit)

 \item \texttt{suck\_u4}: load four \texttt{unsigned byte}s (32 bit)

 \item \texttt{suck\_u8}: load eight \texttt{unsigned byte}s (64 bit)

 \item \texttt{suck\_float}: load four \texttt{byte}s (32 bit)
 converted into a \texttt{float} value

 \item \texttt{suck\_double}: load eight \texttt{byte}s (64 bit)
 converted into a \texttt{double} value

 \item \texttt{suck\_nbytes}: load \textit{n} bytes
\end{itemize}

Loading \texttt{signed} values is done via the
\texttt{suck\_s[1,2,4,8]} macros which cast the loaded bytes to
\texttt{signed} values. All these functions take a
\texttt{classbuffer} (Figure \ref{classbufferstructure}) structure
pointer as argument.

\begin{figure}[h]
\begin{verbatim}
        typedef struct classbuffer {
            classinfo *class;               /* pointer to classinfo structure     */
            u1        *data;                /* pointer to byte code               */
            s4         size;                /* size of the byte code              */
            u1        *pos;                 /* current read position              */
        } classbuffer;
\end{verbatim}
\caption{\texttt{classbuffer} structure}
\label{classbufferstructure}
\end{figure}

This \texttt{classbuffer} structure is filled with data via the

\begin{verbatim}
        classbuffer *suck_start(classinfo *c);
\end{verbatim}

function. This function tries to locate the class, specifed with the
\texttt{classinfo} structure, in the \texttt{CLASSPATH}. This can be
a plain class file in the filesystem or a file in a
\texttt{zip}/\texttt{jar} file. If the class file is found, the
\texttt{classbuffer} is filled with data collected from the class
file, including the class file size and the binary representation of
the class.

Before reading any byte of the binary representation with a
\texttt{suck\_*} function, the remaining bytes in the
\texttt{classbuffer} data array must be checked with the

\begin{verbatim}
        static inline bool check_classbuffer_size(classbuffer *cb, s4 len);
\end{verbatim}

function. If the remaining bytes number is less than the amount of the
bytes to be read, specified by the \texttt{len} argument, a
\texttt{java.lang.ClassFormatError} with the detail message
\textit{Truncated class file}---as mentioned before---is thrown.

The following subsections describe chronologically in greater detail
the individual loading steps of a class or interface from it's binary
representation.


\subsection{Constant pool loading}
\label{sectionconstantpoolloading}

The class' constant pool is loaded via

\begin{verbatim}
        static bool class_loadcpool(classbuffer *cb, classinfo *c);
\end{verbatim}

from the \texttt{constant\_pool} table in the binary representation of
the class of interface. The constant pool needs to be parsed in two
passes. In the first pass the information loaded is saved in temporary
structures, which are further processed in the second pass, when the
complete constant pool has been traversed. Only when the whole
constant pool entries have been loaded, any constant pool entry can be
completely resolved, but this resolving can only be done in a specific
order:

\begin{enumerate}
 \item \texttt{CONSTANT\_Class}

 \item \texttt{CONSTANT\_String}

 \item \texttt{CONSTANT\_NameAndType}

 \item \texttt{CONSTANT\_Fieldref}, \texttt{CONSTANT\_Methodref} and
 \texttt{CONSTANT\_InterfaceMethodref} --- these are combined into one
 structure
\end{enumerate}

\begingroup
\tolerance 10000
The remaining constant pool types \texttt{CONSTANT\_Integer},
\texttt{CONSTANT\_Float}, \texttt{CONSTANT\_Long},
\texttt{CONSTANT\_Double} and \texttt{CONSTANT\_Utf8} can be
completely resolved in the first pass and need no further processing.

\endgroup

The temporary structures, shown in Figure
\ref{constantpoolstructures}, are used to \textit{forward} the data
from the first pass into the second.

\begin{figure}[h]
\begin{verbatim}
        /* CONSTANT_Class entries */
        typedef struct forward_class {
            struct forward_class *next;
            u2 thisindex;
            u2 name_index;
        } forward_class;

        /* CONSTANT_String */
        typedef struct forward_string {
            struct forward_string *next;
            u2 thisindex;
            u2 string_index;
        } forward_string;

        /* CONSTANT_NameAndType */
        typedef struct forward_nameandtype {
            struct forward_nameandtype *next;
            u2 thisindex;
            u2 name_index;
            u2 sig_index;
        } forward_nameandtype;

        /* CONSTANT_Fieldref, CONSTANT_Methodref or CONSTANT_InterfaceMethodref */
        typedef struct forward_fieldmethint {
            struct forward_fieldmethint *next;
            u2 thisindex;
            u1 tag;
            u2 class_index;
            u2 nameandtype_index;
        } forward_fieldmethint;
\end{verbatim}
\caption{temporary constant pool structures}
\label{constantpoolstructures}
\end{figure}

The \texttt{classinfo} structure has two pointers to arrays which
contain the class' constant pool infos, namely: \texttt{cptags} and
\texttt{cpinfos}. \texttt{cptags} contains the type of the constant
pool entry. \texttt{cpinfos} contains a pointer to the constant pool
entry itself. In the second pass the references are resolved and the
runtime structures are created. In further detail this includes for

\begin{itemize}
 \item \texttt{CONSTANT\_Class}: get the UTF8 name string of the
 class, store type \texttt{CONSTANT\_Class} in \texttt{cptags}, create
 a class in the class hashtable with the UTF8 name and store the
 pointer to the new class in \texttt{cpinfos}

 \item \texttt{CONSTANT\_String}: get the UTF8 string of the
 referenced string, store type \texttt{CONSTANT\_String} in
 \texttt{cptags} and store the UTF8 string pointer into
 \texttt{cpinfos}

 \begingroup
 \tolerance 10000
 \item \texttt{CONSTANT\_NameAndType}: create a
 \texttt{constant\_nameandtype} (Figure \ref{constantnameandtype})
 structure, get the UTF8 name and description string of the field or
 method and store them into the \texttt{constant\_nameandtype}
 structure, store type \texttt{CONSTANT\_NameAndType} into
 \texttt{cptags} and store a pointer to the
 \texttt{constant\_nameandtype} structure into \texttt{cpinfos}

 \endgroup

\begin{figure}[h]
\begin{verbatim}
        typedef struct {            /* NameAndType (Field or Method)       */
            utf *name;              /* field/method name                   */
            utf *descriptor;        /* field/method type descriptor string */
        } constant_nameandtype;
\end{verbatim}
\caption{\texttt{constant\_nameandtype} structure}
\label{constantnameandtype}
\end{figure}

 \begingroup
 \tolerance 10000
 \item \texttt{CONSTANT\_Fieldref}, \texttt{CONSTANT\_Methodref} and
 \texttt{CONSTANT\_InterfaceMethodref}: create a
 \texttt{constant\_FMIref} (Figure \ref{constantFMIref}) structure,
 get the referenced \texttt{constant\_nameandtype} structure which
 contains the name and descriptor resolved in a previous step and
 store the name and descriptor into the \texttt{constant\_FMIref}
 structure, get the pointer of the referenced class, which was created
 in a previous step, and store the pointer of the class into the
 \texttt{constant\_FMIref} structure, store the type of the current
 constant pool entry in \texttt{cptags} and store a pointer to
 \texttt{constant\_FMIref} in \texttt{cpinfos}

 \endgroup

\begin{figure}[h]
\begin{verbatim}
        typedef struct {           /* Fieldref, Methodref and InterfaceMethodref    */
            classinfo *class;      /* class containing this field/method/interface  */
            utf       *name;       /* field/method/interface name                   */
            utf       *descriptor; /* field/method/interface type descriptor string */
        } constant_FMIref;
\end{verbatim}
\caption{\texttt{constant\_FMIref} structure}
\label{constantFMIref}
\end{figure}

\end{itemize}

Any UTF8 strings, \texttt{constant\_nameandtype} structures or
referenced classes are resolved with the

\begin{verbatim}
        voidptr class_getconstant(classinfo *c, u4 pos, u4 ctype);
\end{verbatim}

function. This functions checks for type equality and then returns the
requested \texttt{cpinfos} slot of the specified class.


\subsection{Interface loading}

Interface loading is very simple and straightforward. After reading
the number of interfaces, for every interface referenced, a
\texttt{u2} constant pool index is read from the currently loading
class or interface. This index is used to resolve the interface class
via the \texttt{class\_getconstant} function from the class' constant
pool. This means, interface \textit{loading} is more interface
\textit{resolving} than loading. The resolved interfaces are stored
in an \texttt{classinfo *} array allocated by the class loader. The
memory pointer of the array is assigned to the \texttt{interfaces}
field of the \texttt{clasinfo} structure.


\subsection{Field loading}

The number of fields of the class or interface is read as \texttt{u2}
value. For each field the function

\begin{verbatim}
        static bool field_load(classbuffer *cb, classinfo *c, fieldinfo *f);
\end{verbatim}

is called. The \texttt{fieldinfo *} argument is a pointer to a
\texttt{fieldinfo} structure (Figure \ref{fieldinfostructure})
allocated by the class loader. The fields' \texttt{name} and
\texttt{descriptor} are resolved from the class constant pool via
\texttt{class\_getconstant}. If the verifier option is turned on, the
fields' \texttt{flags}, \texttt{name} and \texttt{descriptor} are
checked for validity and can result in a
\texttt{java.lang.ClassFormatError}.

\begin{figure}[h]
\begin{verbatim}
    struct fieldinfo {        /* field of a class                                 */
        s4   flags;           /* ACC flags                                        */
        s4   type;            /* basic data type                                  */
        utf *name;            /* name of field                                    */
        utf *descriptor;      /* JavaVM descriptor string of field                */
	
        s4   offset;          /* offset from start of object (instance variables) */

        imm_union  value;     /* storage for static values (class variables)      */

        classinfo *class;     /* needed by typechecker. Could be optimized        */
                              /* away by using constant_FMIref instead of         */
                              /* fieldinfo throughout the compiler.               */
        ...
    };
\end{verbatim}
\caption{\texttt{fieldinfo} structure}
\label{fieldinfostructure}
\end{figure}

Each field can have some attributes. The number of attributes is read
as \texttt{u2} value from the binary representation. If the field has
the \texttt{ACC\_FINAL} bit set in the flags, the
\texttt{ConstantValue} attribute is available. This is the only
attribute processed by \texttt{field\_load} and can occur only once,
otherwise a \texttt{java.lang.ClassFormatError} is thrown. The
\texttt{ConstantValue} entry in the constant pool contains the value
for the \texttt{final} field. Depending on the fields' type, the
proper constant pool entry is resolved and assigned.


\subsection{Method loading}

As for the fields, the number of the class or interface methods is read from
the binary representation as \texttt{u2} value. For each method the function

\begin{verbatim}
        static bool method_load(classbuffer *cb, classinfo *c, methodinfo *m);
\end{verbatim}

is called. The beginning of the method loading code is nearly the same
as the field loading code. The \texttt{methodinfo *} argument is a
pointer to a \texttt{methodinfo} structure allocated by the class
loader. The method's \texttt{name} and \texttt{descriptor} are
resolved from the class constant pool via
\texttt{class\_getconstant}. With the verifier turned on, some method
checks are carried out. These include \texttt{flags}, \texttt{name}
and \texttt{descriptor} checks and argument count check.

\begin{figure}[h]
\begin{verbatim}
    struct methodinfo {                 /* method structure                       */
        java_objectheader header;       /* we need this in jit's monitorenter     */
        s4          flags;              /* ACC flags                              */
        utf        *name;               /* name of method                         */
        utf        *descriptor;         /* JavaVM descriptor string of method     */
        ...
        bool        isleafmethod;       /* does method call subroutines           */

        classinfo  *class;              /* class, the method belongs to           */
        s4          vftblindex;         /* index of method in virtual function    */
                                        /* table (if it is a virtual method)      */
        s4          maxstack;           /* maximum stack depth of method          */
        s4          maxlocals;          /* maximum number of local variables      */
        s4          jcodelength;        /* length of JavaVM code                  */
        u1         *jcode;              /* pointer to JavaVM code                 */
        ...
        s4          exceptiontablelength;/* exceptiontable length                 */
        exceptiontable *exceptiontable; /* the exceptiontable                     */

        u2          thrownexceptionscount;/* number of exceptions attribute       */
        classinfo **thrownexceptions;   /* checked exceptions a method may throw  */

        u2          linenumbercount;    /* number of linenumber attributes        */
        lineinfo   *linenumbers;        /* array of lineinfo items                */
        ...
        u1         *stubroutine;        /* stub for compiling or calling natives  */
        ...
    };
\end{verbatim}
\caption{\texttt{methodinfo} structure}
\label{methodinfostructure}
\end{figure}

The method loading function has to distinguish between a
\texttt{native} and a ''normal'' JAVA method. Depending on the
\texttt{ACC\_NATIVE} flags, a different stub is created.

For a JAVA method, a \textit{compiler stub} is created. The purpose of
this stub is to call the CACAO jit compiler with a pointer to the byte
code of the JAVA method as argument to compile the method into machine
code. During code generation a pointer to this compiler stub routine
is used as a temporary method call, if the method is not compiled
yet. After the target method is compiled, the new entry point of the
method is patched into the generated code and the compiler stub is
needless, thus it is freed.

If the method is a \texttt{native} method, the loader tries to find
the native function. If the function was found, a \textit{native stub}
is generated. This stub is responsible to manipulate the method's
arguments to be suitable for the \texttt{native} method called. This
includes inserting the \textit{JNI environment} pointer as first
argument and, if the \texttt{native} method has the
\texttt{ACC\_STATIC} flag set, inserting a pointer to the methods
class as second argument. If the \texttt{native} method is
\texttt{static}, the native stub also checks if the method's class is
already initialized. If the method's class is not initialized as the
native stub is generated, a \texttt{asm\_check\_clinit} calling code
is emitted.

Each method can have some attributes. The method loading function
processes two of them: \texttt{Code} and \texttt{Exceptions}.

The \texttt{Code} attribute is a \textit{variable-length} attribute
which contains the Java Virtual Machine instructions---the byte
code---of the JAVA method. If the method is either \texttt{native} or
\texttt{abstract}, it must not have a \texttt{Code} attribute,
otherwise it must have exactly one \texttt{Code}
attribute. Additionally to the byte code, the \texttt{Code} attribute
contains the exception table and attributes to \texttt{Code} attribute
itself. One exception table entry contains the \texttt{start\_pc},
\texttt{end\_pc} and
\texttt{handler\_pc} of the \texttt{try-catch} block, each read as
\texttt{u2} value, plus a reference to the class of the
\texttt{catch\_type}. Currently there are two attributes of the
\texttt{Code} attribute defined in the JVM specification:
\texttt{LineNumberTable} and \texttt{LocalVariableTable}. CACAO only
processes the \texttt{LineNumberTable} attribute. A
\texttt{LineNumberTable} entry consist of the \texttt{start\_pc} and
the \texttt{line\_number}, which are stored in a \texttt{lineinfo}
structure (Figure \ref{lineinfostructure}).

\begin{figure}[h]
\begin{verbatim}
    struct lineinfo {
        u2 start_pc;
        u2 line_number;
    };
\end{verbatim}
\caption{\texttt{lineinfo} structure}
\label{lineinfostructure}
\end{figure}

The linenumber count and the memory pointer of the \texttt{lineinfo}
structure array are assigned to the \texttt{classinfo} fields
\texttt{linenumbercount} and \texttt{linenumbers} respectively.

The \texttt{Exceptions} attribute is a \textit{variable-length}
attribute and contains the checked exceptions the JAVA method may
throw. The \texttt{Exceptions} attribute consist of the count of
exceptions, which is stored in the \texttt{classinfo} field
\texttt{thrownexceptionscount}, and the adequate amount of \texttt{u2}
constant pool index values. The exception classes are resolved from
the constant pool and stored in an allocated \texttt{classinfo *}
array, whose memory pointer is assigned to the
\texttt{thrownexceptions} field of the \texttt{classinfo} structure.

Any attributes which are not processed by the CACAO class loading
system, are skipped via

\begin{verbatim}
        static bool skipattributebody(classbuffer *cb);
\end{verbatim}

which skips one attribute or

\begin{verbatim}
        static bool skipattributes(classbuffer *cb, u4 num);
\end{verbatim}

which skips a specified number \texttt{num} of attributes. If any
problem occurs in the method loading function, a
\texttt{java.lang.ClassFormatError} with a specific detail message is
thrown.


\subsection{Attribute loading}

Attribute loading is done via the

\begin{verbatim}
        static bool attribute_load(classbuffer *cb, classinfo *c, u4 num);
\end{verbatim}

function. The currently loading class or interface can contain some
additional attributes which have not already been loaded. The CACAO
system class loader processes two of them: \texttt{InnerClasses} and
\texttt{SourceFile}.

The \texttt{InnerClass} attribute is a \textit{variable-length}
attribute in the \texttt{attributes} table of the binary
representation of the class or interface. A \texttt{InnerClass} entry
contains the \texttt{inner\_class} constant pool index itself, the
\texttt{outer\_class} index, the \texttt{name} index of the inner
class' name and the inner class' \texttt{flags} bitmask. All these
values are read in \texttt{u2} chunks.

The constant pool indexes are used with the

\begin{verbatim}
        voidptr innerclass_getconstant(classinfo *c, u4 pos, u4 ctype);
\end{verbatim}

function call to resolve the classes or UTF8 strings. After resolving
is done, all values are stored in the \texttt{innerclassinfo}
structure (Figure \ref{innerclassinfostructure}).

\begin{figure}[h]
\begin{verbatim}
    struct innerclassinfo {
        classinfo *inner_class;       /* inner class pointer                      */
        classinfo *outer_class;       /* outer class pointer                      */
        utf       *name;              /* innerclass name                          */
        s4         flags;             /* ACC flags                                */
    };
\end{verbatim}
\caption{\texttt{innerclassinfo} structure}
\label{innerclassinfostructure}
\end{figure}

The other attribute, \texttt{SourceFile}, is just one \texttt{u2}
constant pool index value to get the UTF8 string reference of the
class' \texttt{SourceFile} name. The string pointer is assigned to the
\texttt{sourcefile} field of the \texttt{classinfo} structure.

Both attributes must occur only once. Other attributes than these two
are skipped with the earlier mentioned \texttt{skipattributebody}
function.

After the attribute loading is done and no error occured, the
\texttt{class\_load\_intern} function returns the \texttt{classinfo}
pointer to signal that there was no problem. If \texttt{NULL} is
returned, there was an exception.


\section{Dynamic class loader}


\section{Eager - lazy class loading}

A Java Virtual Machine can implement two different algorithms for the
system class loader to load classes or interfaces: \textit{eager class
loading} and \textit{lazy class loading}.


\subsection{Eager class loading}
\label{sectioneagerclassloading}

The Java Virtual Machine initially creates, loads and links the class
of the main program with the system class loader. The creation of the
class is done via the \texttt{class\_new} function call (see section
\ref{sectionsystemclassloader}). In this function, with \textit{eager
loading} enabled, firstly the currently created class or interface is
loaded with \texttt{class\_load}. CACAO uses the \textit{eager class
loading} algorithm with the command line switch \texttt{-eager}. As
described in the ''Constant pool loading'' section (see
\ref{sectionconstantpoolloading}), the binary representation of a
class or interface contains references to other classes or
interfaces. With \textit{eager loading} enabled, referenced classes or
interfaces are loaded immediately.

If a class reference is found in the second pass of the constant pool
loading process, the class is created in the class hashtable with
\texttt{class\_new\_intern}. CACAO uses the intern function here
because the normal \texttt{class\_new} function, which is a wrapper
function, instantly tries to \textit{link} all referenced
classes. This must not happen until all classes or interfaces
referenced are loaded, otherwise the Java Virtual Machine gets into an
indefinite state.

After the \texttt{classinfo} of the class referenced is created, the
class or interface is \textit{loaded} via the \texttt{class\_load}
function (described in more detail in section
\ref{sectionsystemclassloader}). When the class loading function
returns, the current referenced class or interface is added to a list
called \texttt{unlinkedclasses}, which contains all loaded but
unlinked classes referenced by the currently loaded class or
interface. This list is processed in the \texttt{class\_new} function
of the currently created class or interface after \texttt{class\_load}
returns. For each entry in the \texttt{unlinkedclasses} list,
\texttt{class\_link} is called which finally \textit{links} the class
(described in more detail in section \ref{sectionlinking}) and then
the class entry is removed from the list. When all referenced classes
or interfaces are linked, the currently created class or interface is
linked and the \texttt{class\_new} functions returns.


\subsection{Lazy class loading}
\label{sectionlazyclassloading}

With \textit{eager class loading}, usually it takes much more time for
a Java Virtual Machine to start a program as with \textit{lazy class
loading}. With \textit{eager class loading}, a typical
\texttt{HelloWorld} program needs 513 class loads with the current GNU
classpath CACAO is using. When using \textit{lazy class loading},
CACAO only needs 121 class loads for the same \texttt{HelloWorld}
program. This means with \textit{lazy class loading} CACAO needs to
load more than four times less class files. Furthermore CACAO does
also \textit{lazy class linking}, which saves much more run-time here.

CACAO's \textit{lazy class loading} implementation does not completely
follow the JVM specification. A Java Virtual Machine which implements
\textit{lazy class loading} should load and link requested classes or
interfaces at runtime. But CACAO does class loading and linking at
parse time, because of some problems not resolved yet. That means, if
a Java Virtual Machine instruction is parsed which uses any class or
interface references, like \texttt{JAVA\_PUTSTATIC},
\texttt{JAVA\_GETFIELD} or any \texttt{JAVA\_INVOKE*} instructions,
the referenced class or interface is loaded and linked immediately
during the parse pass of currently compiled method. This introduces
some incompatibilities with other Java Virtual Machines like Sun's
JVM, IBM's JVM or Kaffe.

Imagine a code snippet like this

\begin{verbatim}
        void sub(boolean b) {
            if (b) {
                new A();
            }
            System.out.println("foobar");
        }
\end{verbatim}

If the function is called with \texttt{b} equal \texttt{false} and the
class file \texttt{A.class} does not exist, a Java Virtual Machine
should execute the code without any problems, print \texttt{foobar}
and exit the Java Virtual Machine with exit code 0. Due to the fact
that CACAO does class loading and linking at parse time, the CACAO
Virtual Machine throws an \texttt{java.lang.NoClassDefFoundError:~A}
exception which is not caught and thus discontinues the execution
without printing \texttt{foobar} and exits.

The CACAO development team has not yet a solution for this
problem. It's not trivial to move the loading and linking process from
the compilation phase into runtime, especially CACAO was initially
designed for \textit{eager class loading} and \textit{lazy class
loading} was implemented at a later time to optimize class loading and
to get a little closer to the JVM specification. \textit{Lazy class
loading} at runtime is one of the most important features to be
implemented in the future. It is essential to make CACAO a standard
compliant Java Virtual Machine.


\section{Linking}
\label{sectionlinking}

Linking is the process of preparing a previously loaded class or
interface to be used in the Java Virtual Machine's runtime
environment. The function which performs the linking in CACAO is

\begin{verbatim}
        classinfo *class_link(classinfo *c);
\end{verbatim}

This function, as for class loading, is just a wrapper function for
the main linking function

\begin{verbatim}
        static classinfo *class_link_intern(classinfo *c);
\end{verbatim}

This function should not be called directly and is thus declared as
\texttt{static}. The purposes of the wrapper function are

\begin{itemize}
 \item enter a monitor on the \texttt{classinfo} structure, so that is
 guaranteed that only one thread can link the same class at the same
 time

 \item measure linking time if requested

 \item check if the intern linking function has thrown an error or an
 exception and reset the \texttt{linked} field of the
 \texttt{classinfo} structure

 \item leave the monitor
\end{itemize}

The \texttt{class\_link} function, like the \texttt{class\_load}
function, is implemented to be \textit{reentrant}. This must be the
case for the linking algorithm implemented in CACAO. Furthermore this
means that serveral threads can link different classes or interfaces
at the same time on multiprocessor machines.

The first step in the \texttt{class\_link\_intern} function is to set
the \texttt{linked} field of the currently linked \texttt{classinfo}
structure to \texttt{true}. This is essential, that the linker does
not try to link a class or interface again, while it's already in the
linking process. Such a case can occur because the linker also
processes the class' direct superclass and direct superinterfaces.

In CACAO's linker the direct superinterfaces are processed first. For
each interface in the \texttt{interfaces} field of the
\texttt{classinfo} structure is checked if there occured an
\texttt{java.lang.ClassCircularityError}, which happens when the
currently linked class or interface is equal the interface which
should be processed. Otherwise the interface is loaded and linked if
not already done. After the interface is loaded successfully, the
interface flags are checked for the \texttt{ACC\_INTERFACE} bit. If
this is not the case, a
\texttt{java.lang.IncompatibleClassChangeError} is thrown and
\texttt{class\_link\_intern} returns.

Then the direct superclass is handled. If the direct superclass is
equal \texttt{NULL}, we have the special case of linking
\texttt{java.lang.Object}. There are only set some \texttt{classinfo}
fields to special values for \texttt{java.lang.Object} like

\begin{verbatim}
        c->index = 0;
        c->instancesize = sizeof(java_objectheader);
        vftbllength = 0;
        c->finalizer = NULL;
\end{verbatim}

If the direct superclass is non-\texttt{NULL}, CACAO firstly detects
class circularity as for interfaces. If no
\texttt{java.lang.ClassCircularityError} was thrown, the superclass is
loaded and linked if not already done before. Then some flag bits of
the superclass are checked: \texttt{ACC\_INTERFACE} and
\texttt{ACC\_FINAL}. If one of these bits is set an error is thrown.

If the currently linked class is an array, CACAO calls a special array
linking function

\begin{verbatim}
        static arraydescriptor *class_link_array(classinfo *c);
\end{verbatim}

This function firstly checks if the passed \texttt{classinfo} is an
\textit{array of arrays} or an \textit{array of objects}. In both
cases the component type is created in the class hashtable via
\texttt{class\_new} and then loaded and linked. If none is the case,
the passed array is a \textit{primitive type array}. No matter of
which type the array is, an \texttt{arraydescriptor} structure (Figure
\ref{arraydescriptorstructure}) is allocated and filled with the
appropriate values of the array type.

\begin{figure}[h]
\begin{verbatim}
    struct arraydescriptor {
        vftbl_t *componentvftbl; /* vftbl of the component type, NULL for primit. */
        vftbl_t *elementvftbl;   /* vftbl of the element type, NULL for primitive */
        s2       arraytype;      /* ARRAYTYPE_* constant                          */
        s2       dimension;      /* dimension of the array (always >= 1)          */
        s4       dataoffset;     /* offset of the array data from object pointer  */
        s4       componentsize;  /* size of a component in bytes                  */
        s2       elementtype;    /* ARRAYTYPE_* constant                          */
    };
\end{verbatim}
\caption{\texttt{arraydescriptor} structure}
\label{arraydescriptorstructure}
\end{figure}



\section{Initialization}

