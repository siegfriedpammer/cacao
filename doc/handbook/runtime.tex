\chapter{Run Time System}

\section{Introduction}

\section{Object layout}

\section{Method invocation}

Java and the programming language Theta do not implement multiple
inheritance, but single inheritance with multiple subtyping. This important
difference makes object layout and method dispatch more efficient. Although
the bidirectional layout was designed for a language with multiple
subtyping, it has the problem that more than one {\em vtbl} pointer has to
be included in objects. The CACAO JIT compiler \cite{KrGr97b} moves the
dispatch header of the bidirectional layout into the class information with
negative offsets from the {\em vtbl}. For each implemented interface there
is a distinct interface {\em vtbl}. Unimplemented interfaces are
represented by null pointers.
%
\begin{figure*}[htb]
\begin{center}
\setlength{\unitlength}{1mm}
\begin{picture}(114,72)
\put( 0,42){\vector(1,0){12}}
\put( 0,42){\makebox(8,6){\it objptr}}
\put(12,60){\makebox(24,6){\it object}}
\put(48,63){\makebox(30,6){\it class}}
\put(90,66){\makebox(24,6){\it code}}
\put(12,42){\makebox(24,18){\it instance data}}
\put(12,36){\makebox(24,6){\it class pointer}}
\put(12,42){\line(1,0){24}}
\put(36,39){\vector(1,0){12}}
\put(48,57){\makebox(30,6){\it method pointer}}
\put(48,57){\line(1,0){30}}
\put(48,51){\makebox(30,6){\it method pointer}}
\put(48,51){\line(1,0){30}}
\put(48,45){\makebox(30,6){\it method pointer}}
\put(48,45){\line(1,0){30}}
\put(48,39){\makebox(30,6){\it class info}}
\put(48,27){\makebox(30,6){\it interface pointer}}
\put(48,33){\line(1,0){30}}
\put(48,33){\makebox(30,6){\it interface pointer}}
\put(48, 0){\makebox(30,6){\it method pointer}}
\put(48, 6){\makebox(30,6){\it method pointer}}
\put(48, 6){\line(1,0){30}}
\put(48,21){\makebox(30,6){\it interfaces}}
\put(90,36){\framebox(24,6){\it method code}}
\put(44,15){\vector(1,0){4}}
\put(44,15){\line(0,1){15}}
\put(44,30){\line(1,0){4}}
\put(40, 0){\vector(1,0){8}}
\put(40, 0){\line(0,1){36}}
\put(40,36){\line(1,0){8}}
\put(78, 3){\line(1,0){8}}
\put(86, 3){\line(0,1){51}}
\put(78, 9){\line(1,0){4}}
\put(82, 9){\line(0,1){39}}
\put(78,18){\line(1,0){4}}
\put(78,54){\line(1,0){8}}
\put(86,36){\vector(1,0){4}}
\put(90,48){\framebox(24,6){\it method code}}
\put(78,48){\vector(1,0){12}}
\put(90,60){\framebox(24,6){\it method code}}
\put(78,60){\vector(1,0){12}}
\thicklines
\put(48, 0){\framebox(30,12){}}
\put(48,15){\framebox(30,6){\it method pointer}}
\put(12,36){\framebox(24,24){}}
\put(48,27){\framebox(30,36){}}
\put(48,39){\line(1,0){30}}
\end{picture}
\caption{CACAO object and compact class descriptor layout}
\label{objectcompactlayout}
\end{center}
\end{figure*}

To call a virtual method, two memory access instructions are necessary
(load the class pointer, load the method pointer) followed by the call
instruction. Calling an interface method needs an additional indirection.
%
\begin{figure*}[htb]
\begin{center}
\setlength{\unitlength}{1mm}
\begin{picture}(114,42)
\put(0,15){\vector(1,0){12}}
\put(0,15){\makebox(12,6){\it objptr}}
\put(12,33){\makebox(24,6){\it object}}
\put(48,36){\makebox(30,6){\it class}}
\put(90,39){\makebox(24,6){\it code}}
\put(12,15){\makebox(24,18){\it instance data}}
\put(12,9){\makebox(24,6){\it class pointer}}
\put(12,15){\line(1,0){24}}
\put(36,12){\vector(1,0){12}}
\put(48,30){\makebox(30,6){\it method pointer}}
\put(48,30){\line(1,0){30}}
\put(48,24){\makebox(30,6){\it method pointer}}
\put(48,24){\line(1,0){30}}
\put(48,18){\makebox(30,6){\it method pointer}}
\put(48,18){\line(1,0){30}}
\put(48,12){\makebox(30,6){\it class info}}
\put(48,0){\makebox(30,6){\it ifmethod pointer}}
\put(48,6){\line(1,0){30}}
\put(48,6){\makebox(30,6){\it ifmethod pointer}}
\put(90,3){\framebox(24,6){\it method code}}
\put(78,3){\vector(1,0){12}}
\put(82,3){\line(0,1){18}}
\put(78,21){\line(1,0){4}}
\put(78,27){\line(1,0){8}}
\put(78,9){\line(1,0){8}}
\put(86,9){\line(0,1){18}}
\put(86,18){\vector(1,0){4}}
\put(90,18){\framebox(24,6){\it method code}}
\put(78,33){\vector(1,0){12}}
\put(90,33){\framebox(24,6){\it method code}}
\thicklines
\put(12,9){\framebox(24,24){}}
\put(48,0){\framebox(30,36){}}
\put(48,12){\line(1,0){30}}
\end{picture}
\caption{CACAO object and fast class descriptor layout}
\label{objectfastlayout}
\end{center}
\end{figure*}

In the faster scheme, we store interface methods in an additional table at
negative offsets from the {\em vtbl} (see figure~\ref{objectfastlayout}).
Segregating the interface virtual function table keeps the standard virtual
function table small and allows interface methods to be called with just
two memory accesses. The memory consumption of virtual function tables
containing interface and class methods would be {\em number of (classes +
interfaces) $\times$ number of distinct methods}. The memory consumption of
the interface tables is only {\em number of classes which implement
interfaces $\times$ number of interface methods}. Colouring can be used to
reduce the number of distinct offsets for interface methods further, but
complicates dynamic class loading, leading to renumbering and code
patching.

The Jalapeno virtual machine implements an interface method invocation
similar to the fast class descriptor layout of CACAO, but instead of
colouring hashing of the method indices is used \cite{Alpern+01}. The table
for the interface method pointers is allocated with a fixed size much
smaller than the number of interface methods. When two method indices hash
to the same offset, a conflict resolving stub is called instead of the
interface methods directly. For conflict resolution the stub is passed the
method index in a scratch register as additional argument. An interface
method invocation can be executed with the following four machine
instructions:
%
\begin{verbatim}
LD  vtblptr,(obj)                  ; load vtbl pointer
LD  mptr,hash(method_ptr)(vtblptr) ; load method pointer
MV  mindex,idreg                   ; load method index
JSR (mptr)                         ; call method (or conflict stub)
\end{verbatim}
%
The number of machine instructions is the same as in the compact class
descriptor layout of CACAO, but the indirection is eliminated, which
reduces the number of cycles needed for the execution of this instruction
sequence on a pipelined architecture. Compared to the old interface method
invocation in the Jalapeno VM, which searched the superclass hierarchy for
a matching method signature, the new method resulted in changes in the
run-time from one percent slowdowns to speedups up to 51 percent.


\section{Run time type checking}

Since type tests for trees are more (space) efficient than type tests for
DAGs, a possible solution is to split a DAG into a tree part and the
remaining graph. For languages with single inheritance and multiple
subtyping this partitioning of the class hierarchy is already done in the
language itself.

\begin{figure}[htb]
\begin{center}
\setlength{\unitlength}{1mm}
\begin{picture}(86,26)
\thicklines
\put(23, 3){\circle{6}}
\put(43, 3){\circle{6}}
\put(63, 3){\circle{6}}
\put(33,13){\circle{6}}
\put(53,13){\circle{6}}
\put(43,23){\circle{6}}
\put(20, 0){\makebox(6,6){D}}
\put(40, 0){\makebox(6,6){E}}
\put(60, 0){\makebox(6,6){F}}
\put(30,10){\makebox(6,6){B}}
\put(50,10){\makebox(6,6){C}}
\put(40,20){\makebox(6,6){A}}
\put(19, 0){\makebox(0,6)[r]{\{3,0\}}}
\put(47, 0){\makebox(6,6)[l]{\{4,0\}}}
\put(67, 0){\makebox(0,6)[l]{\{6,0\}}}
\put(29,10){\makebox(0,6)[r]{\{2,2\}}}
\put(57,10){\makebox(0,6)[l]{\{5,1\}}}
\put(47,20){\makebox(0,6)[l]{\{1,5\}}}
\put(25, 5){\line( 1,1){6}}
\put(35,15){\line( 1,1){6}}
\put(41, 5){\line(-1,1){6}}
\put(51,15){\line(-1,1){6}}
\put(61, 5){\line(-1,1){6}}
\end{picture}
\caption{Relative numbering with \{{\tt baseval}, {\tt diffval}\} pairs}
\label{relnumbering}
\end{center}
\end{figure}

CACAO uses a subtype test based on relative numbering for classes and a
kind of matrix implementation for interfaces. Two numbers {\em low} and
{\em high} are stored for each class in the class hierarchy. A depth first
traversal of the hierarchy increments a counter for each class and assigns
the counter to the {\em low} field when the class is first encountered and
assigns the counter to the {\em high} field when the traversal leaves the
class. In languages with dynamic class loading a renumbering of the
hierarchy is needed whenever a class is added. A class {\em sub} is a
subtype of another class {\em super}, if {\em super.low $\le$ sub.low $\le$
super.high}. Since a range check is implemented more efficiently by an
unsigned comparison, CACAO stores the difference between the {\em low} and
{\em high} values and compares it against the difference of the {\em low}
values of both classes. The code for {\tt instanceof} looks similar to:

\begin{verbatim}
return (unsigned) (sub->vftbl->baseval - super->vftbl->baseval) <=
       (unsigned) (super->vftbl->diffval);
\end{verbatim}

For leaf nodes in the class hierarchy the {\tt diffval} is 0 which results
in a faster test (a simple comparison of the {\tt baseval} fields of the
sub- and superclass). In general a JIT compiler can generate the faster
test only for final classes. An AOT compiler or a JIT compiler which does
patching of the already generated machine code may additionally replace
both the {\tt baseval} and the {\tt diffval} of the superclass by a
constant. Currently CACAO uses constants only when dynamic class loading is
not used.

CACAO stores an interface table at negative offsets from the virtual method
table (see figure~\ref{objectcompactlayout}). This table is needed for the
invocation of interface methods. The same table is additionally used by the
subtype test for interfaces. If the table is empty for the index of the
superclass, the subtype test fails. The code for {\tt instanceof} looks
similar to:

\begin{verbatim}
return (sub->vftbl->interfacetable[-super->index] != NULL);
\end{verbatim}

Both subtype tests can be implemented by very few machine code instructions
without using branches which are expensive on modern processors.


\section{Exception handling}

\subsection{Introduction}

Exceptions in Java occur either implicitly or explicitly. Typical
implicit exceptions are references to the {\tt null} pointer, array
index out of bounds and division by zero. Exceptions also can be
raised explicitly with the {\tt throw} instruction. To handle
exceptions occurring during execution, code which can raise an
exception is included in a {\tt try} block. An efficient
implementation of exception handling has to take care of managing {\tt
try} blocks and to check for implicit exceptions efficiently .


\subsection{Known implementation techniques}

Three standard methods exist for implementing exception handling:

\begin{itemize}

\item dynamically create a linked list of {\tt try} block data
      structures,

\item use static {\tt try} block tables and search these tables at run
      time (suggested for JavaVM interpreters),

\item use functions with two return values.

\end{itemize}

The first method has been used in portable implementations of
exception handling for C++ \cite{Cameron+92} or Ada \cite{Giering+94}
using {\tt setjmp} and {\tt longjmp}. A linked exception handling data
structure is created when entering a {\tt try} block and the structure
is discarded when leaving the protected block. Java requires precise
exceptions. It means that all expressions evaluated before the
exception raising instruction must have finished and all expressions
after the raising instruction must not have been started. Therefore,
in practice, some instructions may not be moved freely. In the case of
subroutine calls, the callee-saved registers must be restored to their
original value. The data structure can be used to store such
information. The disadvantage of this method is that creating and
discarding of the data structure takes some time even if an exception
is never raised.

The second method has been suggested for an efficient exception
handling implementation of C++ \cite{KoenigStroustrup90} and is used
in Java implementations. For every method, the JavaVM maintains an
exception table. This exception table contains the program counter of
the start and the end of the {\tt try} block, the program counter of
the exception handler and the type of the exception. A JavaVM
interpreter can easily interpret this structure and dispatch to the
corresponding handler code. If the byte code is translated to native
code, the equivalent technique is more complicated.

To simplify restoration of the registers, the old CACAO implementation
used a different scheme \cite{KrGr97b}. A method has two return
values: the real return value and an exception value stored in a
register. After each method call, the exception register is checked
and, if it is non-zero, the exception handling code is executed. Since
an exception is rarely raised, the branch is easy to predict and
cheap. Entering and leaving a {\tt try} block have no associated
costs. At compile time, the dispatch information contained in the
exception table is translated into method dispatching native code. 

Run time checks for null pointers and array bounds are quite frequent,
but can be eliminated in many cases. It is often possible to move a
loop invariant null pointer check before the loop or to eliminate a
bound check. Some more sophisticated compilers use these code motion
techniques.


\subsection{Motivation for a change}

\begin{table*}
\begin{center}
\begin{tabular}[b]{|l|c|c|c|c|c|}
\hline 
                    & JavaLex & javac & espresso & Toba & java\_cup \\
\hline
null pointer checks &  6859   &  8197 &  11114   & 5825 &   7406    \\
\hline
method calls        &  3226   &  7498 &   7515   & 4401 &   5310    \\
\hline
try blocks          &    20   &   113 &     44   &   28 &     27    \\
\hline
\end{tabular}
\caption{Number of pointer checks, method invocations and try blocks}
\label{MethodTry}
\end{center}
\end{table*}

The old CACAO implementation was simple, but it only makes sense if
the number of try blocks is high. We made an empirical study to count
the numbers of static occurrences of method invocations and of {\tt
try} blocks in some applications (see table \ref{MethodTry}). The
number of method invocations is two magnitudes bigger than the number
of {\tt try} blocks. Furthermore, an exception is rarely raised during
program execution. This led us to a new implementation of exception
handling.


\subsection{The new exception handling scheme}

The new exception handling scheme is similar to that in a JavaVM
interpreter. If an exception occurs, information in the exception
table is interpreted. However native code complicates the matter.

The pointers to Java byte code must be replaced by pointers to native
code. It requires that, during native code generation, the order of
basic blocks not be allowed to change. If basic blocks are eliminated
because of dead code, the information about a block can not be
discarded if there is a pointer to it in the exception table.

A CACAO stack frame only contains copies of saved or spilled
registers. There is no saved frame pointer. The size of a stack frame
is only contained in the instructions which allocate and deallocate
the stack. Therefore, to support exception handling, additional
information has to be stored elsewhere.

The code for a method needs access to constants (mostly address
constants). Since a global constant table would be too large for short
address ranges and, because methods are compiled on demand, every
method has its own constant area which is allocated directly before
the start of the method code (see fig.\ \ref{methodlayout}). A
register is reserved which contains the method pointer. Constants are
addressed relative to the method pointer.

\begin{figure}[htb]
\begin{center}
\setlength{\unitlength}{1mm}
\begin{picture}(50,18)
\put(24,6){\vector(-1,0){6}}
\put(24,3){\makebox(25,6){\it method pointer}}
\put(0,0){\makebox(18, 6){\it constants}}
\put(0,6){\makebox(18,12){\it code}}
\thicklines
\put(0,6){\line(1,0){18}}
\put(0,0){\framebox(18,18){}}
\end{picture}
\caption{CACAO method layout}
\label{methodlayout}
\end{center}
\end{figure}

During a method call, the method pointer of the calling method is
destroyed. However the return address is stored in a register which is
preserved during execution of the called method and has to be used for
returning from the method. After a method return, the method pointer
of the calling method is recomputed using the return address. The
following code for a method call demonstrates the method calling
convention:

\begin{verbatim}
LDQ cp,(obj)     ; load class pointer
LDQ mp,met(cp)   ; load method pointer
JSR ra,(mp)      ; call method
LDA mp=ra+offset ; recompute method pointer
\end{verbatim}

At the beginning of the constant area, there are fields which contain
the necessary information for register restoration:

\begin{mylist}{{\tt framesize}}
\item[{\tt framesize}] the size of the stack frame
\item[{\tt isleaf}]    a flag which is true if the method is a leaf
method
\item[{\tt intsave}]   number of saved integer registers
\item[{\tt floatsave}] number of saved floating point registers
\item[{\tt extable}]   the exception table -- similar to the JavaVM
table
\end{mylist}

The exception handler first checks if the current executing method has
an associated handler and may dispatch to this handler. If there is no
handler, it unwinds the stack and searches the parent methods for a
handler. The information in the constant area is used to restore the
registers and update the stack pointer. The return address and the
offset in the immediately following {\tt LDA} instruction is used to
recompute the method pointer.

\begin{table*}
\begin{center}
\begin{tabular}[b]{|l|c|c|c|c|c|}
\hline 
          & JavaLex & javac  & espresso & Toba  & java\_cup \\ \hline
CACAO old &  61629  & 156907 &  122951  & 67602 &   87489   \\ \hline
CACAO new &  37523  &  86346 &   69212  & 41315 &   52386   \\ \hline
\end{tabular}
\caption{Number of generated native instructions}
\label{CodeSize}
\end{center}
\end{table*}

The change to the new scheme allowed us to implement the null pointer
check for free. We protect the first 64 Kbyte of memory against read
and write access. If an bus error is raised, we catch the signal and
check if the faulting address is within the first 64K. If this is the
case, we dispatch to our exception handler, otherwise we propagate the
signal. This gives faster programs and reduces the work of the
compiler in generating pointer checking code. As shown in table
\ref{MethodTry}, the numbers of null pointer checks are quite high.


\section{Garbage collector}


