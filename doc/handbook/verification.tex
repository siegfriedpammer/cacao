\section{Verification}

\newcommand{\cmdline}[1]{\texttt{#1}}
\newcommand{\srcfile}[1]{\texttt{#1}}
\newcommand{\code}[1]{\texttt{#1}}
\newcommand{\asgn}{\sqsubseteq}
\newcommand{\qed}{$\Box$}
\newcommand{\norm}[1]{\textrm{norm}\left(#1\right)}
\newcommand{\reft}[2]{\langle #1, #2\rangle}
\newtheorem{definition}{Definition}
\newtheorem{lemma}{Lemma}
\newtheorem{theorem}{Theorem}
\newenvironment{proof}{\noindent\emph{Proof.} }{}

Cacao verifies class files according to the JVM
Specification \cite{javavm99} with some exceptions. We will first give
an overview of the verifier implementation and then discuss specific
points where the implementation deviates from the official JVM
Specification.

You can turn off class file verification using the \cmdline{-noverify}
option\footnote{The \cmdline{-noverify} option disables verifier
checks which are especially time-consuming. Checks which can be
performed without significant overhead are always enabled.} when
invoking
\cmdline{cacao}.

\subsection{Implementation Overview}

The verification checks which are concerned with the overall structure
of the class file and the contents of the constant pool (logical
passes 1 and 2 in the specification \cite{javavm99}) reside in
\srcfile{loader.c} (with UTF-8 validation in
\srcfile{tables.c}). These passes of the verifier are executed when a
class file is loaded and linked, before the JIT starts to compile any
method within the loaded class.

The checks concerning the bytecode array of a method (logical passes 3
and 4 in the specification \cite{javavm99}) are performed when the JIT
compiles the method\footnote{Usually when the method is called for the
first time.}. These checks are implemented in
\srcfile{jit/parse.c}, \srcfile{jit/stack.c} and
\srcfile{jit/typecheck.c}. The files \srcfile{typeinfo.[ch]}
implement the type system used by \srcfile{jit/typecheck.c}.

\srcfile{jit/parse.c} contains the following checks which are performed
while translating the Java Bytecode to the intermediate
representation:

\begin{itemize}
\item Detect invalid opcodes,
\item detect unexpected end of bytecode,
\item check if branch targets are within bounds,
\item check if branch targets are valid instruction starts,
\item check constant pool indices and types,
\item check local variable indices (no type check),
\item check exception handler ranges.
\end{itemize}

The code in \srcfile{jit/stack.c} builds a model of the operand stack
and checks the intermediate instructions for type safety regarding the
five basic stack slot types: integer\footnote{The basic \emph{integer}
type is the union of the Java types \code{int}, \code{short},
\code{byte}, \code{char} and \code{boolean}.}, float, long, double and
address. (The basic type \emph{address} encompasses all reference and
\code{returnAddress} types.) Checks performed while building the stack
model include:

\begin{itemize}
\item Detect operand stack underflow,
\item detect operand stack overflow,
\item detect undefined stack depth at merging points,
\item detect undefined basic stack slot types at merging points,
\item check type-safe use of stack slots regarding the five basic
      types.
\end{itemize}

The stack model and the intermediate instructions then enter the type
checker implemented in \srcfile{jit/typecheck.c}. The type checker
uses a special type system to perform the following checks:

\begin{itemize}
\item Check type-safe use of reference types,
\item check type-safe use of local variables,
\item detect invalid use of uninitialized objects,
\item check type-safe method invocations and field access,
\item check local subroutines and use of \code{returnAddress}es,
\item check type-safe use of \code{return} instructions,
\item check access rights.
\end{itemize}

\subsection{Identifiers}

Cacao currently does not check if the names of classes, methods and
fields are valid Java identifiers since this would require including a
Unicode database in Cacao. Instead the following checks are performed:

\begin{itemize}
\item Names must be valid non-empty UTF-8 strings.
\item Names must not contain ASCII control characters or the
      \code{NUL}-character.
\item A method name starting with the ASCII character `\code{<}' must be
      either \code{<init>} or \code{<clinit>}.
\item A \code{NameAndType} entry in the constant pool may not
      reference a method named \code{<clinit>}.
\item Field names may not start with `\code{<}'.

\end{itemize}

\subsection{Reference Types}

The approach for checking reference types described in the JVM
Specification \cite{javavm99} does only work well for single
inheritance since the rule for merging reference types is taking the
least common ancestor of the types. When multiple inheritance via
interfaces is used this can lead to type-safe programs being rejected
by the verifier.

Cacao's rule for merging reference types is equivalent to using set
union instead of the least common ancestor. To optimize for the common
case of trivial merges\footnote{Merges of a class $C$ with itself or
one of its superclasses or subclasses.}, however, the Cacao
implementation uses a slightly more complex scheme which is detailed
below.

\subsubsection{Array Classes}

For the purpose of verification Cacao assumes that array classes have
subclass relationships induced by their component types. For example
\code{Integer[]} is considered a subclass of \code{Number[]}. The
class \code{Object[]} and all primitive arrays are considered to be
subclasses of a (pseudo) class \code{Arraystub} with the following
declaration:

\begin{verbatim}
class Arraystub extends Object implements Cloneable, java.io.Serializable
{
}
\end{verbatim}

\subsubsection{Formal Treatment}

\begin{definition} 
A reference type in Cacao is an ordered pair $\reft{C}{M}$ where $C$
is a class\footnote{In this section the term `class' includes Java
classes, Java interfaces and Java array classes.} and M is a (possibly
empty) set\footnote{Implemented as a sorted list.} of proper
subclasses of $C$. The Cacao type corresponding to a simple Java
class, interface, or array class $C$ is $\reft{C}{\emptyset}$.  
\end{definition}

The following rule is used to check if $\reft{C}{M}$ is assignable
(denoted $\sqsubseteq$) to the class $K$:

\begin{eqnarray*}
\reft{C}{M}\asgn K 
& \Longleftrightarrow &
(C\asgn K) \vee ((M \neq \emptyset) \wedge
(\forall c \in M: c\asgn K)) \\
& \Longleftrightarrow &
\forall c \in \textrm{cls}(\reft{C}{M}): c\asgn K
\end{eqnarray*}

where

\[\textrm{cls}(\reft{C}{M}) = 
\left\{ \begin{array}{ll} \{C\} & \mbox{if $M = \emptyset$} \\
                         M     & \mbox{if $M \neq \emptyset$}
       \end{array}
\right.\]

The rule for merging two reference types is:

\[\reft{C_1}{M_1} \sqcup \reft{C_2}{M_2} = 
\norm{\textrm{lca}(C_1,C_2),\textrm{cls}(\reft{C_1}{M_1}) \cup \textrm{cls}(\reft{C_2}{M_2})}\]

where $\textrm{lca}(C_1,C_2)$ denotes the least common ancestor of
$C_1$ and $C_2$ and

\[\norm{C,M} = 
\left\{ \begin{array}{ll} \reft{C}{\emptyset} & \mbox{if $C \in M$} \\
                          \reft{C}{M}         & \mbox{if $C \notin M$}
       \end{array}
\right.\]

We will now prove that the merging rule is sound in the sense that it
does not lead to the verification of unsafe programs or to the
rejection of type-safe programs.

\begin{lemma}
\label{lemma:sub}
For any class $C$ and any reference type $\reft{C_2}{M_2}$ where $C_2$
is $C$ or a subclass of $C$

\[\reft{C}{\emptyset}\asgn K \Longrightarrow \reft{C_2}{M_2}\asgn K\]

for all classes $K$.
\end{lemma}
\begin{proof}
All classes in $\textrm{cls}(\reft{C_2}{M_2})$ are $C$ or subclasses
of $C$ and thus assignable to $C$. The lemma follows by transitivity.\hfill\qed
\end{proof}

\begin{theorem}
For any two reference types $T_1=\reft{C_1}{M_1}$ and $T_2=\reft{C_2}{M_2}$ the merging rule satisfies

\[((T_1\asgn K) \wedge (T_2\asgn K))
\Longleftrightarrow (T_1 \sqcup T_2) \asgn K\]

for all classes $K$.
\end{theorem}

\begin{proof}
Let $T = T_1 \sqcup T_2 = \reft{C}{M}$ and 
$M_V = \textrm{cls}(T_1) \cup \textrm{cls}(T_2)$.

\noindent\textit{Case 1: $C \notin M_V$.} Because of $M = M_V \neq
\emptyset$ we have $\textrm{cls}(T) = M_V$ and the
theorem becomes

\[\forall K:
((\forall c \in \textrm{cls}(T_1): c\asgn K)
\wedge
(\forall c \in \textrm{cls}(T_2): c\asgn K))
\Longleftrightarrow
\forall c \in \textrm{cls}(T_1) \cup \textrm{cls}(T_2): c\asgn K\]

which is true by definition of $\cup$.

\noindent\textit{Case 2: $C \in M_V$.} We have $C =
\textrm{lca}(C_1,C_2)$ and $M = \emptyset$. Because all elements of
$M_1$ and $M_2$ are proper subclasses of $C_1$ and $C_2$ respectively
we know that $C \notin M_1 \cup M_2$. From this and $C \in M_V$
follows that $C \in \{C_1\}\cup\{C_2\}$ and so at least one of $M_1$,
$M_2$ must be $\emptyset$. Without reduction of generality let $M_1 =
\emptyset$. If $C = C_1$ we get

\[
((\reft{C}{\emptyset}\asgn K) \wedge (\reft{C_2}{M_2}\asgn K))
\Longleftrightarrow
\reft{C}{\emptyset}\asgn K
\]

\noindent From left to right this is trivial and from right to left it
follows from Lemma~\ref{lemma:sub}.

\noindent If $C \neq C_1$ then $C = C_2$ and $M_2 = \emptyset$
(because of $C \in M_V = \{C_1\}\cup\textrm{cls}(T_2)$) and we get:

\[
((\reft{C_1}{\emptyset}\asgn K) \wedge (\reft{C}{\emptyset}\asgn K))
\Longleftrightarrow
\reft{C}{\emptyset}\asgn K
\]

\noindent which is proven in the same way.\hfill\qed

\end{proof}


\subsection{Local Subroutines}

Cacao uses a very different scheme for type checking local subroutines
than that implied by the JVM Specification. We will now discuss the
reasons for this deviation and explain why the official specification
should be considered flawed with respect to local subroutines (see
also \cite{Coglio01}).

Type checking local subroutines tends to be the most complex part of
bytecode verification. This arises from the requirement \cite{javavm99}
that subroutines be polymorphic over the types of local variables that
they do not use\footnote{Intuitively this means that if a subroutine
does not use a certain local variable this variable will maintain its
type across any \code{jsr} `call' of that subroutine.}. This
requirement cannot be met by a simple data flow analysis which would
merge the types of all local variables entering a subroutine via
different \code{jsr} instructions.

The JVM Specification suggests an implementation of the verifier which
tries to model a kind of call stack for local subroutines. The problem
with this approach -- apart from the severe restrictions it inflicts
on subroutines -- is that the behavior of exceptions is undefined with
respect to this (fictitious) call stack. When an exception can be
thrown at a certain point in a subroutine it is impossible for the
verifier to determine whether the responsible handler should be
considered to be `inside' or `outside' the subroutine. Thus the
fundamental flaw of this approach is the implicit assumption that
subroutines have well-defined boundaries, which is not the
case\footnote{At least it is not guaranteed by the documented
constraints on the bytecode and the semantics of \code{jsr} and
\code{ret}.}.

Cacao has therefore adopted an alternative algorithm for type checking
local subroutines. The algorithm is formally described and proven to
be sound in \cite{Coglio02}. We will now give an informal explanation
of the algorithm's idea and discuss a certain restriction which the
Cacao implementation introduces in addition to those in
\cite{Coglio02}.

The basic idea is to treat the \code{jsr} instruction like any other
control flow instruction while avoiding the `lossy' merging of types.
We denote as $(s,v)$ a \emph{pair}, where $s$ is a list of types
corresponding to the stack slots and $v$ is a vector of types
corresponding to the local variables at a certain bytecode
instruction. A simple data flow analysis would map every position in
the bytecode to a single pair. The enhanced algorithm maps every
position in the bytecode to a non-empty set of pairs representing the
alternative possible `assignments' of types to the stack slots and
local variables at this position. When control flow is merged at a
certain point we just compute the union of the sets of pairs which
are propagated from the predecessors of the merging point. This way
when we are type checking a subroutine we can maintain all possible
types that a (possibly unused) variable can have `inside' the
subroutine.

The second important measure is typing \code{returnAddress}es by the
bytecode address they return to. This means if we encounter a
\code{jsr} instruction we assign the stack top a type $\code{ret}_p$
where $p$ is the address of the instruction following the \code{jsr}.
Thus when we reach a $\code{ret}\ i$ instruction we just have to look
at the types assigned to the variable $i$ in each pair for the current
instruction\footnote{The type of variable $i$ must be a type of the
form $\code{ret}_x$ in each pair, otherwise the bytecode is invalid.} 
and we know the target addresses that the \code{ret} can reach and
which pairs propagate to which target address.

The Cacao implementation differs from the algorithm in \cite{Coglio02}
in one point: Cacao always merges stack slot types (with the exception
of \code{returnAddress}es) when merging control flow. In other words:
Cacao does not keep a set of pairs $(s,v)$, instead it keeps a single
list of types\footnote{Stored directly in the stack model.} $s$ for the
stack and a set of vectors\footnote{See the
\code{typevector\_} and \code{typevectorset\_} functions in
\srcfile{typeinfo.c}.} $\{v_1, v_2, \dots\}$ for the local variables.
\code{returnAddress}es on the stack contain a special list which keeps a
$\code{ret}_{x_i}$ type for every vector $v_i$ in the set of local
variable vectors. This design was chosen in order to keep the impact
of subroutine verification on the stack model and the code dealing
with stack operations minimal. The consequence of this is that
subroutines may not be polymorphic with respect to stack slot
types. This is a restriction when compared to the original algorithm
in \cite{Coglio02} but it is in agreement with the JVM Specification,
which requires that at any instruction in the bytecode the stack depth
and the types of the stack slots must be well-defined.
