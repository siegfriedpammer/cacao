\section{IA32 (x86, i386) code generator}

Porting to the famous x86 platform was more effort than
expected. CACAO was designed to run on RISC machines from ground up,
so the whole code generation part has to be adapted. The first
approach was to replace the simple RISC macros with x86 code, but this
turned out to be not successful. So new x86 code generation macros
were written, with no respect to the old RISC macros.

Some smaller problems occured since the x86 port was the first 32 bit
target platform, like segmentation faults due to heap corruption,
which turned out to be a simple \texttt{for} loop bug only hit on 32
bit systems. Most of the CACAO system already was
\textit{32-bit-ready}, namely an architecture dependent
\texttt{types.h} with definitions of the used datatypes and some
feature flags, which features the processor itself natively
supports. Most noticeable change was the \texttt{s8} and \texttt{u8}
datatype, changed from \texttt{long} to \texttt{long long} to support
64 bit calculations.


\subsection{Code generation}

One big difference in writing the new code generation macros was, that
the x86 architecture is not a \textit{load-store architecture} like
the RISC machines, but the \textit{machine instructions} can handle
both \textit{memory operands} and \textit{register operands}. This led
to a much more complicated handling of the various ICMDs. The typical
handling of an ICMD on RISC machines looks like this (on the example
of the integer add ICMD):

\begin{verbatim}
        case ICMD_IADD:
            var_to_reg_int(s1, src->prev, REG_ITMP1);
            var_to_reg_int(s2, src, REG_ITMP2);
            d = reg_of_var(iptr->dst, REG_ITMP3);
            M_IADD(s1, s2, d);
            store_reg_to_var_int(iptr->dst, d);
            break;
\end{verbatim}

This means loading the two \textit{source operands} from memory to
temporary registers, if necessary, getting a \textit{destination
register}, do the calculation and store the result to memory, if the
destination variable resides in memory. If all operands are assigned
to registers, only the calculation is done. This design also works on
x86 machines but leads to much bigger code size, reduces decoding
bandwith and increases register pressure in the processor itself,
which results in lower performance \ref{}. Thus we use all kinds of
instruction types that are available and decide which one we have to
use in some \texttt{if} statements:

\begin{verbatim}
        if (IS_INMEMORY(iptr->dst)) {
            if (IS_INMEMORY(src) && IS_INMEMORY(src->prev)) {
                ...
            } else if (IS_INMEMORY(src) && !IS_INMEMORY(src->prev)) {
                ...
            } else if (!IS_INMEMORY(src) && IS_INMEMORY(src->prev)) {
                ...
            } else {
                ...
            }
        } else {
            if (IS_INMEMORY(src) && IS_INMEMORY(src->prev)) {
                ...
            } else if (IS_INMEMORY(src) && !IS_INMEMORY(src->prev)) {
                ...
            } else if (!IS_INMEMORY(src) && IS_INMEMORY(src->prev)) {
                ...
            } else {
                ...
            }
        }
\end{verbatim}

For most ICMDs we can further optimize the generated code when one
source variable and the destination variable share the same local
variable.

To be backward compatible, mostly in respect of embedded systems, all
generated code can be run on i386 systems.

Another problem was the access to the functions data segment. Since
RISC platforms like ALPHA and MIPS have a procedure pointer register,
for the x86 platform there had to be implemented a special handling
for accesses to the data segment, like \texttt{PUTSTATIC} and
\texttt{GETSTATIC} instructions. The solution is like the handling of
\textit{jump references} or \textit{check cast references}, which also
have to be code patched, when the code and data segment are
relocated. This means, there has to be an extra
\textit{immediate-to-register} move (\texttt{i386\_mov\_imm\_reg()})
before every \texttt{PUT}/\texttt{GETSTATIC} instruction, which moves
the start address of the procedure, and thus the start address of the
data segment, in one of the temporary registers (code snippet from
\texttt{PUTSTATIC}):

\begin{verbatim}
        i386_mov_imm_reg(0, REG_ITMP2);
        dseg_adddata(mcodeptr);
\end{verbatim}

The \texttt{dseg\_adddata()} call inserts the current postion of the
code generation pointer into a datastructure which is later processed
by \texttt{codegen\_finish()}, where the final address of the data
segment is patched.


\subsection{Constant handling}

Unlike RISC machines the x86 architecture has \textit{immediate move}
instructions which can handle the maximum bitsize of the
registers. Thus we don't have to load big constants indirect from the
data segment, which means a \textit{memory load} instruction, but we
can move 32 bit constants \textit{inline} into their destination
registers.

\begin{verbatim}
        i386_mov_imm_reg(0xcafebabe, REG_ITMP1);
\end{verbatim}

For constants bigger than 32 bits up to 64 bits, we split the move
up into two immediate move instructions.


\subsection{Calling conventions}

The normal calling convention of the x86 processor is passing all
function arguments on the stack. The store size depends on the data
type (the following types reflect the JAVA data types):

\begin{itemize}
 \item \texttt{boolean}, \texttt{byte}, \texttt{char}, \texttt{short}, \texttt{int},
       \texttt{float}, \texttt{void} --- 4 bytes
 \item \texttt{long}, \texttt{double} --- 8 bytes
\end{itemize}

We changed this convention for CACAO in a way, that we are using
always 8 bytes on the stack for each datatype. After calling the function

\begin{verbatim}
        void sub(int i, long l, float f, double d);
\end{verbatim}

we have a stack layout like in figure \ref{stacklayout}.

\begin{figure}[htb]
\begin{center}
\setlength{\unitlength}{1mm}
\begin{picture}(50,54)
\thicklines
\put(0,0){\framebox(24,54){}}
\put(0,42){\line(1,0){24}}
\put(0,36){\line(1,0){24}}
\put(0,30){\line(1,0){24}}
\put(0,18){\line(1,0){24}}
\put(0,12){\line(1,0){24}}
\put(0,6){\line(1,0){24}}
\put(30,0){\vector(-1,0){6}}
\put(30,3){\makebox(24,6){\textit{+4 bytes}}}
\put(30,-3){\makebox(24,6){\textit{stack pointer}}}

\put(0,45){\makebox(24,6){\textit{double value}}}
\put(0,36){\makebox(24,6){\textit{unused}}}
\put(0,30){\makebox(24,6){\textit{float value}}}
\put(0,21){\makebox(24,6){\textit{long value}}}
\put(0,12){\makebox(24,6){\textit{unused}}}
\put(0,6){\makebox(24,6){\textit{int value}}}
\put(0,0){\makebox(24,6){\textit{return address}}}
\end{picture}
\caption{CACAO x86 stack layout after function call}
\label{stacklayout}
\end{center}
\end{figure}

If we pass a 32 bit variable, we just push 4 bytes onto the stack and
leave the remaining 4 bytes untouched. This makes no problems since we
do not read a 64 bit value from a 32 bit location. Passing a 64 bit
value is straightforward.

With this adaptation, it was possible to use the \textit{stack space
allocation algorithm} without any changes. The drawback of this
decision was, that we have to copy all arguments of a native function
call into a new stack frame and we have a slightly bigger memory
footprint.

But calling a native function always means a stack manipulation,
because you have to put the \textit{JNI environment}, and additionally
for \texttt{static} functions the \textit{class pointer}, in front of
the function parameters. So this negligible.

For some \texttt{BUILTIN} functions there had to be written
\texttt{asm\_} counterparts, which copy the 8 byte parameters in their
correct size in a new stack frame. But this only affected
\texttt{BUILTIN} functions with more than 1 function parameter. To be
exactly, 2 functions, namely \texttt{arrayinstanceof} and
\texttt{newarray}. So this is not a big speed impact.

Return parameters are stored in different places, this depends on the
return type of the function:

\begin{itemize}
 \item \texttt{boolean}, \texttt{byte}, \texttt{char}, \texttt{short},
 \texttt{int}, \texttt{void}: return value resides in \texttt{\%eax}
 (\texttt{REG\_RESULT})

 \item \texttt{long}: return value is split up onto the register pair
 \texttt{\%edx:\%eax}
 (\texttt{REG\_RESULT2:REG\_RESULT}, high 32 bit in
 \texttt{\%edx}, low 32 bit in \texttt{\%eax})

 \item \texttt{float}, \texttt{double}: return value resides in the
 \textit{top of stack} element of the \textit{floating point unit}
 stack (\texttt{st(0)}, described in detail later)
\end{itemize}


\subsection{Register allocator}

Register usage was another problem in porting the CACAO to x86. An x86
processor has 8 genernal purpose registers (GPR), of which one is the
\textit{stack pointer} (SP) and thus it can not be used for arithmetic
instructions. From the remaining 7 registers, in \textit{worst-case
instructions} like \texttt{CHECKCAST} or \texttt{INSTANCEOF}, we need
to reserve 3 temporary registers. So we have 4 registers available.

We use \texttt{\%ebp}, \texttt{\%esi}, \texttt{\%edi} as callee saved
registers (which are callee saved registers in the x86 ABI) and
\texttt{\%ebx} as scratch register (which is also a callee saved
register in the x86 ABI, but we need some scratch registers). So we
have a lack of scratch registers. But for most ICMD instructions, we
do not need all, or sometimes none, of the temporary registers.

This fact we use in the \texttt{analyse\_stack()} pass. We try to use
\texttt{\%edx} (which is \texttt{REG\_ITMP3}) and \texttt{\%ecx} (which
is \texttt{REG\_ITMP2}) as scratch registers for the register
allocator if certain ICMD instructions are not used in the compiled
method. So for \textit{best-case situations} CACAO has 3
\textit{callee saved} and 3 \textit{scratch} registers.

This analysis should be changed from \textit{method level} to
\textit{basic-block level}, so CACAO could emit better code on x86.

The register allocator itself is very straightforward, this means, it
does neither \textit{linear scan} nor any other analyse of the methods
variables, but allocates registers for the local variables in order as
they are defined. This may result in good code on RISC machines, as
there are almost always enough registers available, with 32 registers,
but can produce really bad code on x86 processors.

So the first step to make the x86 port more competitive with SUN's or
IBM's JVM would be to rewrite the register allocator.

Basically the allocation order of the register allocator is as
follows:

\begin{itemize}
 \item interface register allocation
 \item scratch register allocation
 \item local register allocation
\end{itemize}

The only change which had to be made to all allocator passes, was the
handling of \texttt{long} variables, because they are all spilled to
memory (described in more detail in \ref{LongArithmetic}).


\subsection{Long arithmetic}\label{LongArithmetic}

Unlike the PowerPC port, we cannot put \texttt{long}'s in two 32 bit
integer registers, since we have to little of them. Maybe this could
bring a speedup, if the register allocator would be more intelligent
or in leaf functions which have only \texttt{long} variables. But this
is not implemented yet. So, the current approach is to store all
\texttt{long}'s in memory, this means they are always spilled.

Nearly all \texttt{long} instructions are inline, except two of them:
\texttt{LDIV} and \texttt{LREM}. These two are computed via
\texttt{BUILTIN} calls. It would be definitely possible to also
inline them, but the code size is too big and the latency is so high,
that the function calls are negligible.

The x86 processor has some machine instructions which are specifically
designed for 64 bit operations. Some of them are

\begin{itemize}
 \item \texttt{cltd} --- Convert Signed Long to Signed Double Long
 \item \texttt{adc} --- Integer Add With Carry
 \item \texttt{sbb} --- Integer Subtraction With Borrow
\end{itemize}

Thus some of the 64 bit calculations like \texttt{LADD} or
\texttt{LSUB} could be executed in two instructions, if both
operand would reside in registers. But this does not apply to CACAO,
yet.

All of the \texttt{long} instructions operate on 64 bit, even if it is
not necessary. The dependency information that would be needed to just
operate on the lower or upper half of the \texttt{long} variable, is
not generated by CACAO.


\subsection{Floating point arithmetic}

Since the i386, with it's i387 counterpart or the i486, the x86
processor has a \textit{floating point unit} (FPU). This FPU is
implemented as a stack with 8 elements (see table \ref{FPUStack}).

\begin{table*}
\begin{center}
\begin{tabular}[b]{|l|l|}
\hline 
st(x) & FPU Data Register Stack \\ \hline
0     & TOS (Top Of Stack) \\ \hline
1     & \\ \hline
2     & \\ \hline
3     & \\ \hline
4     & \\ \hline
5     & \\ \hline
6     & \\ \hline
7     & \\ \hline
\end{tabular}
\caption{x87 FPU Data Register Stack}
\label{FPUStack}
\end{center}
\end{table*}

This stack is designed to wrap around if values are loaded to the
\textit{top of stack} (TOS). For this reason, it has a special register which
points to the TOS. This pointer is increased if a load instruction to
the TOS is executed and decreased for a store from the TOS.

At first sight, the stack design of the FPU is perfect for the stack
based design of a \textit{java virtual machine} (JVM). But there is a
big problem. The JVM stack has no fixed size, so it can grow up to
much more than 8 elements and we get an stack wrap around and thus an
stack overflow. For this reason we need to implement an
\textit{stack-element-to-register-mapping}.

A very basic design idea, is to define a pointer to the current TOS
offset (\texttt{fpu\_st\_offset}). With this offset we can determine
the current register position in the FPU stack of an arbitrary
register.  From the 8 stack elements we need to reserve the last two
ones (\texttt{st(6), st(7)}), so we can load two memory operands onto
the stack and do the arithmetic on them. Most x86 floating point
arithmetic operations have an \textit{do arithmetic and pop one
element} version of the instruction, that means the float arithmetic
is done and the TOS element is poped off. The remaining stack element
has the result of the calculation. On the example of the \texttt{FADD}
ICMD with two memory operands, it looks like this:

\begin{verbatim}
var_to_reg_flt(s1, src->prev, REG_FTMP1); /* load 1st operand in st(0), increase
                                             fpu_st_offset                          */
var_to_reg_flt(s2, src, REG_FTMP2);       /* load 2nd operand in st(0), increase
                                             fpu_st_offset                          */
i386_faddp();       /* add 2 upper most elements st(1) = st(1) + st(0) -- pop st(0) */
fpu_st_offset--;                        /* decrease fpu_st_offset from previous pop */
store_reg_to_var_flt(iptr->dst, d); /* store result -- decrease fpu_st_offset       */
\end{verbatim}

This mapping works very good with \textit{scratch registers}
only. Defining \textit{callee saved float registers} makes some
problemes since the x86 ABI has no callee saved float registers. This
would need a special handling in the \textit{native stub} of a native
function, namely saving the registers and cleaning the whole FPU
stack, because a C function expects a clear FPU stack.

Basically the x86 FPU can handle 3 float data types:

\begin{itemize}
 \item single-precision (32 bit)
 \item double-precision (64 bit)
 \item double extended-precision (80 bit)
\end{itemize}

The FPU has a 16 bit \textit{control register} which has a
\textit{precision control field} (PC) and a \textit{rounding control
field} (RC), each of 2 bit length (see table \ref{PCField} and
\ref{RCField}).

\begin{table*}
\begin{center}
\begin{tabular}[b]{|l|c|}
\hline 
Precision                          & PC Field \\ \hline
single-precision (32 bit)          & 00B      \\ \hline
reserved                           & 01B      \\ \hline
double-precision (64 bit)          & 10B      \\ \hline
double extended-precision (80 bit) & 11B      \\ \hline
\end{tabular}
\caption{Precision Control Field (PC)}
\label{PCField}
\end{center}
\end{table*}

\begin{table*}
\begin{center}
\begin{tabular}[b]{|l|c|}
\hline 
Rounding Mode                 & RC Field \\ \hline
round to nearest (even)       & 00B      \\ \hline
round down (toward -$\infty$) & 01B      \\ \hline
round up (toward +$\infty$)   & 10B      \\ \hline
round toward zero (truncate)  & 11B      \\ \hline
\end{tabular}
\caption{Rounding Control Field (RC)}
\label{RCField}
\end{center}
\end{table*}

The internal data type used by the FPU is always the \textit{double
extended-precision} (80 bit) format. Therefore implementing a IEEE 754
compliant floating point code on x86 processors is not trivial.

Correct rounding to \textit{single-precision} or
\textit{double-precision} is only done if the value is stored into
memory. This means for certain instructions, like \texttt{FMUL} or
\texttt{FDIV}, a special technique called \textit{store-reload}, has
to be implemented. This technique is in fact very simple but takes two
memory accesses more for this instruction.

For single-precision floats the \textit{store-reload} code could looks
like this:

\begin{verbatim}
i386_fstps_membase(REG_SP, 0);    /* store single-precision float to stack  */
i386_flds_membase(REG_SP, 0);     /* load single-precision float from stack */
\end{verbatim}

Another technique which has to be implemented to be IEEE 754
compliant, is \textit{precision mode switching}. Mode switching on the
x86 processor is made with the \texttt{fldcw} (load control word)
instruction. A \texttt{fldcw} instruction has a quite large overhead,
so frequently mode switches should be avoided. For this technique
there are two different approaches:

\begin{itemize}
 \item \textbf{Mode switch on float arithmetic} --- switch the FPU on
 initialization in one precision mode, mostly \textit{double-precision
 mode} because \texttt{double} arithmetic is more common. With this
 setting \texttt{doubles} are calculated correctly. To handle
 \texttt{floats} in this approach, the FPU has to be switched into
 \textit{single-precision mode} before each \texttt{float} instruction
 and switched back afterwards. Needless to say, that this is only
 useful, if \texttt{float} arithmetic is sparse. For a simple
 calculation like

 \begin{verbatim}
        float f = 1.234F;
        float g = 2.345F;
        float e = f * f + g * g;
 \end{verbatim}        

 the generated ICMD's look like this (with comments where precision
 mode switches take place):

 \begin{verbatim}
        ...
        <fpu in double-precision mode>
        FLOAD         1
        FLOAD         1
        <switch fpu to single-precision mode>
        FMUL         
        <switch fpu to double-precision mode>
        FLOAD         2
        FLOAD         2
        <switch fpu to single-precision mode>
        FMUL         
        <switch fpu to double-precision mode>
        <switch fpu to single-precision mode>
        FADD         
        <switch fpu to double-precision mode>
        FSTORE        3
        ...
 \end{verbatim}

 For this corner case situation the mode switches are extrem, but the
 example should demonstrate how this technique works.

 \item \textbf{Mode switch on float data type change} --- switch the
 FPU on initialization in on precision mode, like before, in
 \textit{double-precision mode}. But the difference on this approach
 is, that the precision mode is only switched if the float data type
 is changed. That means if your calculation switches from
 \texttt{double} arithmetic to \texttt{float} or backwards. This
 technique makes much sense due to the fact that there are always a
 bunch of instructions of the same data type in one row in a normal
 program. Now the same example as before with this approach:

 \begin{verbatim}
        ...
        <fpu in double-precision mode>
        FLOAD         1
        FLOAD         1
        <switch fpu to single-precision mode>
        FMUL         
        FLOAD         2
        FLOAD         2
        FMUL         
        FADD         
        FSTORE        3
        ...
 \end{verbatim}

 After this code sequence the FPU is in \textit{single-precision mode}
 and if a function return would occur, the caller function would not
 know in which mode the FPU is currently. One solution would be to
 reset the FPU to \textit{double-precision} on a function return, if
 the actual mode is \textit{single-precision}.
\end{itemize}

These techniques and further researches into optimizations which could
be done, are described in \cite{OgKoNa02}.

All of these described FPU techniques have been implemented in CACAO's
x86 port, but the results were not completly IEEE 754 compliant. So
the CACAO developer team decided to be on the safe side and to store
all float variables in memory, until we have found a solution which is
fast and 100\% compliant.


\subsection{Exception handling}

The exception handling for the x86 architecture is implemented as
intended to be for CACAO. To handle the common and unexpected, but
often checked, \texttt{NullPointerException} very fast, we use
\textit{hardware null-pointer checking}. That means we install a
signal handler for the \texttt{SIGSEGV} operating system signal and in
the handler we forward the exception to CACAO's internal exception
handling system. So if an instruction tries to access the memory at
address \texttt{0x0}, a \texttt{SIGSEGV} signal is raised because the
memory page is not read or writeable. After the signal is hit, we have
to reinstall the handler, so we can catch further exceptions and this
is done in the handler itself.

The \texttt{SIGSEGV} handler is used on any architecture to which
CACAO has been ported. Additionally we install a handler for the
\texttt{SIGFPE} on the x86 architecture. With this handler we can
catch \texttt{ArithmeticException}'s for integer \textit{/ by zero} in
hardware and there is no need to write a helper function which checks
the operands, as it has to be done for the ALPHA or MIPS port.
