\section{AMD64 (x86\_64) code generator}

\subsection{Introduction}

The AMD64~\cite{AMD64} architecture, formerly know as x86\_64, is an
improvement of the Intel IA32 architecture by AMD---Advanced Micro
Devices~\cite{AMD}. The extraordinary success of the IA32 architecture
and the upcoming memory address space problem on IA32 high-end
servers, led to a special design decision by AMD. Unlike Intel, with
it's completely new designed 64-bit architecture---IA64---AMD decided
to extend the IA32 instruction set with new a 64-bit instruction mode.

Due to the fact that the IA32 instructions have no fixed length, like
this is the fact on RISC machines, it was easy for AMD to introduce a
new \textit{prefix byte} called \texttt{REX}. The \textit{REX prefix}
enables the 64-bit operation mode of the following instruction in the
new \textit{64-bit mode} of the processor.

A processor of the AMD64 architecture has two main operating modes:

\begin{itemize}
\item Long Mode
\item Legacy Mode
\end{itemize}

In the \textit{Legacy Mode} the processor acts like an IA32
processor. Any 32-bit operating system or software can be run on these
type of processors without changes, so companies running IA32 servers
and software can change their hardware to AMD64 and their systems are
still operational. This was the main intention for AMD to develop this
architecture. Furthermore the \textit{Long Mode} is split into two
coexistent operating modes:

\begin{itemize}
\item 64-bit Mode
\item Compatibility Mode
\end{itemize}

The \textit{64-bit Mode} exposes the power of this architecture. Any
memory operation now uses 64-bit addresses and ALU instructions can
operate on 64-bit operands. Within \textit{Compatibility Mode} any
IA32 software can be run under the control of 64-bit operating
system. This, as mentioned before, is yet another point for companies
to change their hardware to AMD64. So their software can be slowly
migrated to the new 64-bit systems, but not every type of software is
faster in 64-bit code. Any memory address fetched or stored into
memory needs to transfer now 64-bits instead of 32-bits. This means
twice as much memory transfer as on IA32 machines.

Another crucial point to make the AMD64 architecture faster than IA32,
is the limited number of registers. Any IA32 architecture, from the
early \textit{i386} to the newest generation of \textit{Intel Pentium
4} or \textit{AMD Athlon}, has only 8 general-purpose registers. With
the \textit{REX prefix}, AMD has the ability to increase the amount of
accessible registers by 1 bit. This means in \textit{64-bit Mode} 16
general-purpose registers are available. The value of a \textit{REX
prefix} is in the range \texttt{40h} through \texttt{4Fh}, depending
on the particular bits used (see table \ref{REX}).

\begin{table}
\begin{center}
\begin{tabular}[b]{|c|c|l|}
\hline
Mnemonic & Bit Position & Definition \\ \hline
-        & 7-4          & 0100 \\ \hline
REX.W    & 3            & 0 = Default operand size \\
         &              & 1 = 64-bit operand size \\ \hline
REX.R    & 2            & 1-bit (high) extension of the ModRM \textit{reg} field, \\
         &              & thus permitting access to 16 registers. \\ \hline
REX.X    & 1            & 1-bit (high) extension of the SIB \textit{index} field, \\
         &              & thus permitting access to 16 registers. \\ \hline
REX.B    & 0            & 1-bit (high) extension of the ModRM \textit{r/m} field, \\
         &              & SIB \textit{base} field, or opcode \textit{reg} field, thus \\
         &              & permitting access to 16 registers. \\ \hline
\end{tabular}
\caption{REX Prefix Byte Fields}
\label{REX}
\end{center}
\end{table}


\subsection{Code generation}

AMD64 code generation is mostly the same as on IA32. All new 64-bit
instructions can handle both \textit{memory operands} and
\textit{register operands}, so there is no need to change the
implementation of the IA32 ICMDs.

Much better code generation can be achieved in the area of
\textit{long arithmetic}. Since all 16 general-purpose registers can
hold 64-bit integer values, there is no need for special long
handling, like on IA32 were we stored all long varibales in memory. As
example a simple \texttt{ICMD\_LADD} on IA32, best case shown for
AMD64 --- \texttt{src->regoff == iptr->dst->regoff}:

\begin{verbatim}
        i386_mov_membase_reg(REG_SP, src->prev->regoff * 8, REG_ITMP1);
        i386_alu_reg_membase(I386_ADD, REG_ITMP1, REG_SP, iptr->dst->regoff * 8);
        i386_mov_membase_reg(REG_SP, src->prev->regoff * 8 + 4, REG_ITMP1);
        i386_alu_reg_membase(I386_ADC, REG_ITMP1, REG_SP, iptr->dst->regoff * 8 + 4);
\end{verbatim}

First memory operand is added to second memory operand which is at the
same stack location as the destination operand. This means, there are
four instructions executed for one \texttt{long} addition. If we would
use registers for \texttt{long} variables we could get a
\textit{best-case} of two instructions, namely \textit{add} followed
by an \textit{adc}. On AMD64 we can generate one instruction for this
addition:

\begin{verbatim}
        x86_64_alu_reg_reg(X86_64_ADD, src->prev->regoff, iptr->dst->regoff);
\end{verbatim}

This means, the AMD64 port is \textit{four-times} faster than the IA32
port (maybe even more, because we do not use memory accesses). Even if
we would implement the usage of registers for \texttt{long} variables
on IA32, the AMD64 port would be at least twice as fast.

To be able to use the new 64-bit instructions, we need to prefix
nearly all instructions---some instructions can be used in their
64-bit mode without escaping---with the mentioned \textit{REX prefix}
byte. In CACAO we use a macro called

\begin{verbatim}
        x86_64_emit_rex(size,reg,index,rm)
\end{verbatim}

to emit this byte. The names of the arguments are respective to their
usage in the \textit{REX prefix} itself (see table \ref{REX}).

The AMD64 architecture introduces also a new addressing method called
\textit{RIP-relative addressing}. In 64-bit mode, addressing relative
to the contents of the 64-bit instruction pointer (program counter)
--- called \textit{RIP-relative addressing} or \textit{PC-relative
addressing} --- is implemented for certain instructions. In this
instructions, the effective address is formed by adding the
displacement to the 64-bit \texttt{RIP} of the next instruction. With
this addressing mode, we can replace the IA32 method of addressing
data in the method's data segment. Like in the
\texttt{ICMD\_PUTSTATIC} instruction, the IA32 code

\begin{verbatim}
        a = dseg_addaddress(&(((fieldinfo *) iptr->val.a)->value));
        i386_mov_imm_reg(0, REG_ITMP2);
        dseg_adddata(mcodeptr);
        i386_mov_membase_reg(REG_ITMP2, a, REG_ITMP2);
\end{verbatim}

can be replaced with the new \textit{RIP-relative addressing} code

\begin{verbatim}
        a = dseg_addaddress(&(((fieldinfo *) iptr->val.a)->value));
        x86_64_mov_membase_reg(RIP, -(((s8) mcodeptr + 7) - (s8) mcodebase) + a, REG_ITMP2);
\end{verbatim}

So we can save one instruction on the read or write of an static
variable. The additional offset of \texttt{+ 7} is the code size of
the instruction itself. The fictive register \texttt{RIP} is defined
with

\begin{verbatim}
        #define RIP    -1
\end{verbatim}

Thus we can determine the special \textit{RIP-relative addressing}
mode in the code generating macro
\texttt{x86\_64\_emit\_membase(basereg,disp,dreg)} with

\begin{verbatim}
        if ((basereg) == RIP) {
            x86_64_address_byte(0,(dreg),RBP);
            x86_64_emit_imm32((disp));
            break;
        }
\end{verbatim}

and generate the \textit{RIP-relative addressing} code. As shown in
the code sample, it's an special encoding of the \textit{address byte}
with \texttt{mod} field set to zero and \texttt{RBP} (\texttt{\%rbp})
as baseregister.


\subsection{Constant handling}

As on IA32, the AMD64 code generator can use \textit{immediate move}
instructions to load integer constants into their destination
registers. The 64-bit extensions of the AMD64 architecture can also
load 64-bit immediates inline. So loading a \texttt{long} constant
just uses one instruction, despite of two instructions on the IA32
architecture. Of course the AMD64 code generator uses the \textit{move
long} (\texttt{movl}) instruction to load 32-bit \texttt{int}
constants to minimize code size. The \texttt{movl} instruction clears
the upper 32-bit of the destination register.

\begin{verbatim}
        case ICMD_ICONST:
                ...
                x86_64_movl_imm_reg(cd, iptr->val.i, d);
                ...

        case ICMD_LCONST:
                ...
                x86_64_mov_imm_reg(cd, iptr->val.l, d);
                ...
\end{verbatim}

\texttt{float} and \texttt{double} values are loaded from the data
segment via the \textit{move doubleword or quadword} (\texttt{movd})
instruction with \textit{RIP-relative addressing}.


\subsection{Calling conventions}

The AMD64 calling conventions are described here
\cite{AMD64ABI}. CACAO uses a subset of this calling convention, to
cover its requirements. CACAO just needs to pass the JAVA data types
to called functions, no other special features are required. The byte
sizes of the JAVA data types on the AMD64 port are shown in table
\ref{javadatatypesizes}.

\begin{table}
\begin{center}
\begin{tabular}[b]{|l|c|}
\hline
JAVA Data Type   & Bytes \\ \hline
\texttt{boolean} & 1     \\
\texttt{byte}    &       \\
\texttt{char}    &       \\ \hline
\texttt{short}   & 2     \\ \hline
\texttt{int}     & 4     \\
\texttt{float}   &       \\ \hline
\texttt{long}    & 8     \\
\texttt{double}  &       \\
\texttt{void}    &       \\ \hline
\end{tabular}
\caption{JAVA Data Type sizes on AMD64}
\label{javadatatypesizes}
\end{center}
\end{table}

\subsubsection{Integer arguments}

The AMD64 architecture has 6 integer argument registers. The order of
the argument registers is shown in table
\ref{amd64integerargumentregisters}.

\begin{table}
\begin{center}
\begin{tabular}[b]{|l|l|}
\hline
Register       & Argument Register \\ \hline
\texttt{\%rdi} & 1$^{\rm st}$      \\ \hline
\texttt{\%rsi} & 2$^{\rm nd}$      \\ \hline
\texttt{\%rdx} & 3$^{\rm rd}$      \\ \hline
\texttt{\%rcx} & 4$^{\rm th}$      \\ \hline
\texttt{\%r8}  & 5$^{\rm th}$      \\ \hline
\texttt{\%r9}  & 6$^{\rm th}$      \\ \hline
\end{tabular}
\caption{AMD64 Integer Argument Register}
\label{amd64integerargumentregisters}
\end{center}
\end{table}

As on RISC machines, the remaining integer arguments are passed on the
stack. Each integer argument, regardless of which integer JAVA data
type, uses 8 bytes on the stack.

Integer return values of any integer JAVA data type are stored in
\texttt{REG\_RESULT}, which is \texttt{\%rax}.

\subsubsection{Floating-point arguments}

The AMD64 architecture has 8 floating point argument registers, namely
\texttt{\%xmm0} through \texttt{\%xmm7}. \texttt{\%xmm} registers are
128-bit wide floating point registers on which SSE and SSE2
instructions can operate. Remaining floating point arguments are
passed, like integer arguments, on the stack using 8 bytes per
argument, regardless to the floating-point JAVA data type.

Floating point return values of any floating-point JAVA data type are
stored in \texttt{\%xmm0}.

As shown, the calling conventions for the AMD64 architecture are
similar to the calling conventions of RISC machines, which allows to
use CACAOs \textit{register allocator algorithm} and \textit{stack
space allocation algorithm} without any changes.

Calling native functions means register moves and stack copying like
on RISC machines. This depends on the count of the arguments used for
the called native function. For non-static native functions the first
integer argument has to be the JNI environment variable, so any
arguments passed need to be shifted by one register, which can include
creating a new stackframe and storing some arguments on the
stack. Additionally for static native functions the class pointer of
the current objects' class is passed in the 2$^{\rm nd}$ integer
argument register. This means that the integer argument registers need
to be shifted by two registers.

One difference of the AMD64 calling conventions to RISC type machines,
like Alpha or MIPS, is the allocation of integer and floating point
argument registers with mixed integer and floating point
arguments. Assume a function like this:

\begin{verbatim}
        void sub(int a, float b, long c, double d);
\end{verbatim}

On a RISC machine, like Alpha, we would have an argument register
allocation like in figure \ref{alphaargumentregisterusage}.
\texttt{a?} represent integer argument registers and \texttt{fa?}
floating point argument registers.

\begin{figure}[htb]
\begin{center}
\setlength{\unitlength}{1mm}
\begin{picture}(60,35)
\thicklines
\put(0,15){\framebox(15,10){a0 = a}}
\put(30,15){\framebox(15,10){a2 = c}}
\put(15,5){\framebox(15,10){fa1 = b}}
\put(45,5){\framebox(15,10){fa3 = d}}
\put(0,0){\line(0,1){30}}
\end{picture}
\caption{Alpha argument register usage for \texttt{void sub(int a, float b, long c, double d);}}
\label{alphaargumentregisterusage}
\end{center}
\end{figure}

On AMD64 the same function call would look like in figure
\ref{amd64argumentregisterusage}.

\begin{figure}[htb]
\begin{center}
\setlength{\unitlength}{1mm}
\begin{picture}(60,35)
\thicklines
\put(0,15){\framebox(15,10){a0 = a}}
\put(15,15){\framebox(15,10){a1 = c}}
\put(0,5){\framebox(15,10){fa0 = b}}
\put(15,5){\framebox(15,10){fa1 = d}}
\put(0,0){\line(0,1){30}}
\end{picture}
\caption{AMD64 argument register usage for \texttt{void sub(int a, float b, long c, double d);}}
\label{amd64argumentregisterusage}
\end{center}
\end{figure}

The register assigment would be \texttt{a0 = \%rdi}, \texttt{a1 =
\%rsi}, \texttt{fa0 = \%xmm0} and \texttt{fa1 = \%xmm1}. This special
usage of the argument registers required some changes in the argument
register allocation algorithm for function calls during stack
analysis and some changes in the code generator itself.


\subsection{Register allocator}

As mentioned in the introduction, the AMD64 architecture has 16
general-purpose registers and 16 floating-point registers. One
general-purpose register is reserved for the \textit{stack
pointer}---namely \texttt{\%rsp}---and thus cannot be used for
arithmetic instructions. The register usage as used in CACAO is shown
in table \ref{amd64registerusage}.

\begin{table}
\begin{center}
\begin{tabular}{l|l|l}
Register       & Usage                                         & Callee-saved \\ \hline
\texttt{\%rax} & return register, reserved for code generator  & no           \\
\texttt{\%rcx} & 4$^{\rm th}$ argument register                & no           \\
\texttt{\%rdx} & 3$^{\rm rd}$ argument register                & no           \\
\texttt{\%rbx} & temporary register                            & no           \\
\texttt{\%rsp} & stack pointer                                 & yes          \\
\texttt{\%rbp} & callee-saved register                         & yes          \\
\texttt{\%rsi} & 2$^{\rm nd}$ argument register                & no           \\
\texttt{\%rdi} & 1$^{\rm st}$ argument register                & no           \\
\texttt{\%r8}  & 5$^{\rm th}$ argument register                & no           \\
\texttt{\%r9}  & 6$^{\rm th}$ argument register                & no           \\
\texttt{\%r10}-\texttt{\%r11} & reserved for code generator    & no           \\
\texttt{\%r12}-\texttt{\%r15} & callee-saved register          & yes          \\
\texttt{\%xmm0} & 1$^{\rm st}$ argument register, return register & no        \\
\texttt{\%xmm1}-\texttt{\%xmm7} & argument registers           & no           \\
\texttt{\%xmm8}-\texttt{\%xmm10} & reserved for code generator & no           \\
\texttt{\%xmm11}-\texttt{\%xmm15} & temporary registers        & no           \\
\end{tabular}
\caption{AMD64 Register usage in CACAO}
\label{amd64registerusage}
\end{center}
\end{table}

There is only one change to the original AMD64 \textit{application
binary interface} (ABI). CACAO uses \texttt{\%rbx} as temporary
register, while the AMD64 ABI uses the \texttt{\%rbx} register as
callee-saved register. So CACAO needs to save the \texttt{\%rbx}
register when a JAVA method is called from a native function, like a
JNI function. This is done in \texttt{asm\_calljavafunction} located in
\texttt{jit/x86\_64/asmpart.S}.

In adapting the register allocator there was a problem concerning the
order of the integer argument registers. The order of the first four
argument register is inverted. This fact can be seen in table
\ref{amd64registerusage} which is ordered ascending by the processors'
internal register numbers. That means the ascending search algorithm
for argument registers in the register allocator would allocate the
first four argument registers in the wrong direction. So there is a
little hack implemented in CACAOs register allocator to handle this
fact. After searching the register definition array for the argument
registers, the first four argument registers are interchanged in their
array. This is done by a simple code sequence (taken from
\texttt{jit/reg.inc}):

\begin{verbatim}
        /* 
         * on x86_64 the argument registers are not in ascending order 
         * a00 (%rdi) <-> a03 (%rcx) and a01 (%rsi) <-> a02 (%rdx)
         */
        n = r->argintregs[3];
        r->argintregs[3] = r->argintregs[0];
        r->argintregs[0] = n;

        n = r->argintregs[2];
        r->argintregs[2] = r->argintregs[1];
        r->argintregs[1] = n;
\end{verbatim}


\subsection{Floating-point arithmetic}

The AMD64 architecture has implemented two sets of floating-point
instructions:

\begin{itemize}
\item x87 (i387)
\item SSE/SSE2
\end{itemize}

The x87 \textit{floating-point unit} (FPU) implementation is
completely compatible to the IA32 implementation, since the i386 with
its i387 coproccessor, with all the advantages and drawbacks, like the
8 slot FPU stack.

The SSE/SSE2 technique is taken from the newest generation of Intel
processors, introduced with Intel's Pentium 4, and can process scalar
32-bit \texttt{float} values and scalar 64-bit \texttt{double} values
in the 128-bit wide \texttt{xmm} floating-point registers. While SSE
instructions operate on 32-bit \texttt{float} values, SSE2 is
responsible for 64-bit \texttt{double} values. In CACAO we implemented
the JAVA floating-point instructions using SSE/SSE2, because SSE/SSE2
is much easier to use and should be the technique of the future. In
some areas SSE/SSE2 is slower than the old x87 implementation, even on
the new designed AMD64 architecture, but SSE/SSE2 offers 16
floating-point registers, which should speed up daily JAVA
floating-point calculations. Another big advantage of SSE/SSE2 to x87
is the missing \textit{single-double precision-rounding} problem, as
described in detail in the ``IA32 code generator'' section. With
SSE/SSE2 the 32-bit \texttt{float} and 64-bit \texttt{double}
arithmetic is calculated and rounded completely IEEE 754 compliant, so
no further adjustments need to take place to fullfil JAVAs
floating-point requirements.

In floating-point value to integer value conversions a JVM has to
check for corner cases as described in the JVM specification. This is
done via a simple inline integer compare of the integer result value
and a call to special assembler wrapper functions for builtin calls,
like \texttt{asm\_builtin\_f2i} for \texttt{ICMD\_F2I} ---
\texttt{float} to \texttt{int} conversion. These corner cases are then
computed in a builtin C function with respect to all special cases
like \textit{Infinite} or \textit{NaN} values.


\subsection{Exception handling}

Since the AMD64 architecture is just an extension to the IA32
architecture, an AMD64 processor itself raises the same signals as an
IA32 processor, so we can catch the same signals in our own signal
handlers. This includes the signals \texttt{SIGSEGV} and
\texttt{SIGFPE}.

When a signal of this type is raised and the signal hits our signal
handler, we reinstall the handler, create a new exception object and
jump to a---in assembler---written exception handling code. The
difference to the exception handling code of RISC machines, is the
fact that RISC machines have a \textit{procedure vector} (PV)
register. So it's easy to find the methods' data segment, which starts
at the PV growing down to smaller addresses like a stack. For the IA32
and AMD64 architecture we had to implement a \textit{method tree}
which contains the start \textit{program counter} (PC) and the end PC
for every single JAVA method compiled in CACAO, to find for any
exception PC the corresponding method and thus the PV. We need the
data segment for the methods' exception table (for a detailed
description see section ''Exception handling'').

We use \texttt{SIGSEGV} for \textit{hardware null-pointer checking},
so we can handle this common exception as fast as possible in
CACAO. The signal handler creates a
\texttt{java.lang.NullPointerException}.

\texttt{SIGFPE} is used to catch integer division by zero exceptions
in hardware. The signal handler generates a
\texttt{java.lang.ArithmeticException} with \texttt{/ by zero} as detail
message.

Both exceptions are handled in hardware by default, but they can also
be catched in software when using CACAOs commandline switch
\texttt{-softnull}. On the RISC ports only the \textit{null-pointer
exception} is checked in software when using this switch, but on IA32
and AMD64 both are checked, \texttt{SIGSEGV} and \texttt{SIGFPE}.


\subsection{Related work}

The AMD64 architecture is a reasonably young architecture, released in
April 2003. At the writing of this document the only available 64-bit
operating systems for AMD64 are GNU/Linux---from different
distributors---, FreeBSD, NetBSD and OpenBSD. Microsoft Windows is not
available yet, although it was announced to be released in the first
half of 2004.

The first available 64-bit JVM for the AMD64 architecture was GCC's
GCJ---The GNU Compiler for the Java Programming
Language~\cite{GCJ}. \texttt{gcj} itself is a portable, optimizing,
ahead-of-time compiler for the JAVA Programming Language, which can
compile:

\begin{itemize}
\item JAVA source code directly to native machine code
\item JAVA source code to JAVA bytecode (class files)
\item JAVA bytecode to native machine code
\end{itemize}

One part of the GCJ is \texttt{gij}, which is the JVM
interpreter. Much of the porting effort for the \textit{GNU Compiler
Collection} to the AMD64 architecture was done by people working at
SUSE~\cite{SUSE}.

Long time no AMD64 JIT was available, till Sun~\cite{Sun} released
their AMD64 version of J2SE 1.4.2-rc1 for GNU/Linux by
Blackdown~\cite{Blackdown} in December 2003. At this time our AMD64
JIT was already working for months, but we were not able to release
CACAO, because of the common status of CACAO to be a compliant
JVM. The Sun JVM uses the HotSpot Server VM by default, the HotSpot
Client VM is not available for AMD64 at this time.

The Kaffe~\cite{Wilkinson:97} JVM has ported their interpreter to the
AMD64 architecture for GNU/Linux, but they still have no plans to port
their JIT.
